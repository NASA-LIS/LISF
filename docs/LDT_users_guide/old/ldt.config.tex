\comment{
# This is the full ldt.config file.  It contains all the user-configurable
# options plus documentation.
#
# Please add any updates to the LDT code regarding configuration options
# to this file -- including documentation.
#
# Documentation must be placed in between
#    #latex: BEGIN_DESCRIPTION
#    #latex: END_DESCRIPTION
# markers.
#
# Lines in the ldt.config file must be placed in between
#    #latex: BEGIN_CONFIG
#    #latex: END_CONFIG
# markers.
#
# Documentation for development-only configuration options should be placed
# in between
#    #latex: BEGIN_DEVELOPMENT_ONLY
#    #latex: END_DEVELOPMENT_ONLY
# markers.
#
# All documentation must the marked up with "#latex: " tags.
#
# Actual lines of the ldt.config file should not be marked up.
#
#
# To include this file in the Users' Guide:
# 1) Checkout the latest copy of this file from the repository.
# 2) Place it with the source for the Users' Guide.
# 3) Rename it ldt.config.tex
# 4) Edit the ldt.config.tex file:
#    Remove the string "#latex:"
#    Replace the string "BEGIN_CONFIG" with the string "\begin{Verbatim}"
#    Replace the string "END_CONFIG" with the string "\end{Verbatim}"
#    Replace the string "BEGIN_DESCRIPTION" with the empty string
#    Replace the string "END_DESCRIPTION" with the empty string
#    For the developer's version:
#       Replace the string "BEGIN_DEVELOPMENT_ONLY" with the empty string
#       Replace the string "END_DEVELOPMENT_ONLY" with the empty string
#    For the public version:
#       Delete the lines between the "BEGIN_DEVELOPMENT_ONLY" and
#       "END_DEVELOPMENT_ONLY" strings
#
# These are the commands (vi) that I use to process this file for the
# Users' Guide.
#
#:%s/#latex://
#:%s/BEGIN_DESCRIPTION//
#:%s/END_DESCRIPTION//
#:%s/BEGIN_CONFIG/\\begin{Verbatim}[frame=single]/
#:%s/END_CONFIG/\\end{Verbatim}/
#
# For developer's version
#:%s/BEGIN_DEVELOPMENT_ONLY//
#:%s/END_DEVELOPMENT_ONLY//
# For public version, 1) replace <CR> with the single keycode ^M,
# 2) copy this into register c, and 3) execute it
#/BEGIN_DEVELOPMENT_ONLY<CR>ma/END_DEVELOPMENT_ONLY<CR>mb:'a,'bdelete<CR>@c
#
}


 
 \section{LDT Config File} \label{sec:ldtconfigfile}
 This section describes the options in the \file{ldt.config} file.

 Not all options described here are available in the public
 version of LDT.
 

 
 \subsection{Overall driver options} \label{ssec:driveropts}
 

 
 \var{LDT running mode:} specifies the running mode used in LDT.
 Acceptable values are:

 \begin{tabular}{ll}
 Value & Description  \\
 ``LSM parameter processing''    & LSM Parameter Processing Option \\
 ``DA preprocessing''            & Data Assimilation Preprocessing
                                   Option \\
 ``Ensemble restart processing'' & Deriving an ensemble restart file
                                   Option \\
 ``Restart preprocessing''       & LSM Restart File Preprocessing
                                   Option \\
 ``Metforce processing''         & Meteorological forcing processing
                                   Option (similar to LIS) \\
 ``Metforce temporal downscaling'' & Meteorological forcing temporal
                                   downscaling \\
 ``Statistical downscaling of met forcing'' &  Statistical options to 
                                   downscale or generate forcing climatologies \\
 \end{tabular}
 

 \begin{Verbatim}[frame=single]
LDT running mode:             "LSM parameter processing"
 \end{Verbatim}

 
 \var{Processed LSM parameter filename:} specifies the output filename
 (with netcdf extension) of the
 LSM parameters processed in LDT to go into LIS.
 See a sample lis\_input.d01.nc (Appendix~\ref{sec:ldt_output_format})
 file for a complete specification description.
 

 \begin{Verbatim}[frame=single]
Processed LSM parameter filename:    ./lis_input.d01.nc
 \end{Verbatim}

 
 \var{LIS number of nests:} specifies the number of nests used
 for the run.
 Values 1 or higher are acceptable. The maximum number of nests is
 limited by the amount of available memory on the system.
 The specifications for different nests are done using white spaces
 as the delimiter. Please see below for further explanations. Note
 that all nested domains should run on the same projection and same
 land surface model.
 

 \begin{Verbatim}[frame=single]
LIS number of nests:            1
 \end{Verbatim}

 
 \var{Number of surface model types:} specifies the number of 
 surface model types selected for the LIS simulation.
 Acceptable values are 1 or higher.
 

 \begin{Verbatim}[frame=single]
Number of surface model types:   1 
 \end{Verbatim}

 
 \var{Surface model types:} specifies the names of the surface
 model types. Options include (but to be expanded later):

 \begin{tabular}{ll}
 Value   & Description                \\
 LSM     &  Land surface model type   \\
 Openwater &  Openwater surface type  \\
 \end{tabular}
 

 \begin{Verbatim}[frame=single]
Surface model types:  "LSM"
 \end{Verbatim}

 
 \var{Land surface model:} specifies the land surface model to be run.
  Need to select the model you want to run in LIS-7, so the appropriate
  model parameters are included in the output netcdf file for LIS.
 Acceptable values are:

 \begin{tabular}{ll}
 Value       & Description         \\
 none        & Template LSM        \\
 Noah.2.7.1  & Noah 2.7.1          \\
 Noah.3.2    & Noah 3.2            \\
 Noah.3.3    & Noah 3.3            \\
 Noah.3.6    & Noah 3.6            \\
 Noah-MP.3.6 & Noah-MP 3.6         \\
 CLM.2       & CLM version 2.0     \\
 VIC.4.1.1   & VIC 4.1.1           \\
 VIC.4.1.2   & VIC 4.1.2           \\
 Mosaic      & Mosaic              \\
 HySSIB      & HySSIB              \\
 CLSMF2.5    & Catchment, Fortuna 2.5 \\
 SAC.3.5.6   & Sacramento          \\
 SNOW17      & Snow17              \\
 RDHM.3.5.6  & Sacramento+snow17   \\
 GeoWRSI.2   & GeoWRSI, v2.0       \\
 \end{tabular}
 

 \begin{Verbatim}[frame=single]
Land surface model:       Noah.3.3
 \end{Verbatim}

 
 \var{Lake model:} specifies the lake model type used in a LIS run.
 Currently, only the FLake lake model is incorporated in LIS,
 and both LDT and LIS are set up for additional support of lake
 model installation and development.  For now, the option ``none''
 is recommended.
 

 \begin{Verbatim}[frame=single]
Lake model:               none
 \end{Verbatim}

 
 \var{Routing model:} specifies the river routing model used in a LIS run.
 Currently, only the HYMAP routing scheme parameters are supported in LDT.
 

 \begin{Verbatim}[frame=single]
Routing model:             HYMAP
 \end{Verbatim}

 
 \var{Water fraction cutoff value:} specifies what gridcell
 fraction is to be represented by water (e.g., 0.6 would be 60\%
 covered by water pixels).  This value acts as a threshold in
 determining which gridcell will be included as a water
 or land point (used also in deriving the land/water mask). \\
 

 \begin{Verbatim}[frame=single]
Water fraction cutoff value:     0.5
 \end{Verbatim}

 
 \var{Number of met forcing sources:} specifies the
 number of met forcing datasets to be used. Acceptable
 values are 0 or higher.
 

 \begin{Verbatim}[frame=single]
Number of met forcing sources:  1
 \end{Verbatim}

 
 \var{Met forcing sources:} specifies the meteorological 
 forcing data sources used for a LIS run.

 For more information about LIS's meteorological forcing
 data reader options, please see the 
 ``Land Information System (LIS) Users' Guide'' for more
 information.

 Acceptable values for the sources are:

 \begin{tabular}{ll}
 Value                 & Description                                \\
 ``NONE''              & none                                       \\
 ``CMAP''              & CMAP precipitation fields                  \\
 ``CPC CMORPH''        & CMORPH precipitation fields                \\
 ``ECMWF''             & ECMWF near-realtime analysis               \\
 ``ECMWF reanalysis''  & ECMWF reanalysis(II),
                         based on Berg et al.(2003)                 \\
 ``GDAS''              & GDAS near-realtime analysis                \\
 ``GEOS''              & NASA-GEOS (v3-5) forcing analysis          \\
 ``GEOS5 forecast''    & GEOS v5 forecast fields                    \\
 ``GFS''               & NCEP-GFS forecast fields                   \\
 ``GLDAS''             & Coarse-scale GLDAS-1 forcing               \\
 ``GSWP1''             & GSWP1 forcing                              \\
 ``GSWP2''             & GSWP2 forcing                              \\
 ``MERRA\-Land''       & NASA's MERRA-Land reanalysis               \\
 ``MERRA2''            & NASA's GMAO MERRA2 reanalysis              \\
 ``NAM242''            & NCEP-NAM 242 resolution (Alaska)           \\
 ``NARR''              & North American Regional Reanalysis         \\
 ``NLDAS1''            & NLDAS1 analysis fields                     \\
 ``NLDAS2''            & NLDAS2 analysis fields                     \\
 ``PRINCETON''         & Global Princeton long-term forcing records \\
 ``RFE2(daily)''       & CPC Daily Rainfall estimator fields        \\
 ``RFE2(gdas)''        & CPC RFE2 rainfall adjusted with GDAS/CMAP  \\
                       & precipitation                              \\
 ``CHIRPS2''           & UCSB CHIRPS v2.0 precipitation dataset     \\
 ``CPC STAGEII''       & CPC Stage II radar-based rainfall          \\
 ``CPC STAGEIV''       & CPC Stage IV radar-based rainfall          \\
 ``TRMM 3B42RTV7''     & TRMM-based 3B42 real-time rainfall         \\
 ``TRMM 3B42V6''       & TRMM-based 3B42 V6 rainfall                \\
 ``TRMM 3B42V7''       & TRMM-based 3B42 V7 rainfall                \\
 \end{tabular}
 

 \begin{Verbatim}[frame=single]
Met forcing sources:       "NLDAS2"
 \end{Verbatim}

 
 \var{Blending method for forcings:} specifies the
 blending method to combine forcings if more than one 
 forcing dataset is used. User-entry activated only when
 the ``Metforce processing'' run mode is selected.

 Acceptable values are:

 \begin{tabular}{ll}
 Value    & Description                                    \\
 overlay  & datasets are overlaid on top of each other     \\
          & in the order they are specified                \\
 ensemble & each forcing dataset is assigned to a separate \\
          & ensemble member (option not available yet in LDT). \\
 \end{tabular}
 

 \begin{Verbatim}[frame=single]
Blending method for forcings: overlay
 \end{Verbatim}

 
 \var{Met spatial transform methods:}
 specifies the type of spatial transform or interpolation
 scheme to apply to the forcing dataset(s) selected.
 Acceptable values are:

 \begin{tabular}{ll}
 Value & Description                                    \\
 ``average''           & Upscale via averaging          \\
 ``neighbor''          & Nearest neighbor scheme        \\
 ``bilinear''          & Bilinear interpolation scheme  \\
 ``budget-bilinear''   & Conservative interpolation scheme (``conserves'' quantities) \\
 \end{tabular}

 Bilinear interpolation uses 4 neighboring points to compute the
 interpolation weights. The conservative approach uses 25 neighboring
 points.  This option is designed to conserve water, like for
 precipitation.
 Also, nearest neighbor can be used, which may
 better preserve large pixels (e.g., 1x1 deg), if needed. ``Average''
 can also be selected if upscaling from finer-scale meteorological
 fields (e.g., going from 4 KM to 0.25 deg).
 

 \begin{Verbatim}[frame=single]
Met spatial transform methods:     bilinear
 \end{Verbatim}

 
 \var{Topographic correction method (met forcing):} specifies whether
 to use elevation correction on select forcing fields.
 Acceptable values are:

 \begin{tabular}{ll}
 Value & Description                                               \\
 ``none''        & Do not apply topographic correction for forcing \\
 ``lapse-rate''  & Use lapse rate correction for forcing           \\
 \end{tabular}

 Current meteorological forcing datasets supported for applying this
 lapse-rate adjustment to the temperature, humidity, pressure and 
 downward longwave fields, include:
   NLDAS1, NLDAS2, NAM242, GDAS, PRINCETON, and ECMWF
 Future forcing dataset options will include: GEOS, GLDAS, GFS,
 ECMWF\_reanalysis, and possible others.

  ECMWF and GDAS forcing types include several terrain height maps
 and not just one overall, either due to change in versions or
 gridcell size, respective.
 

 \begin{Verbatim}[frame=single]
Topographic correction method (met forcing):  "lapse-rate"
 \end{Verbatim}

 
 \var{Temporal interpolation method (met forcing):} 
 specifies the type of temporal interpolation scheme to 
 apply to the met forcing data.
 Acceptable values are:

 \begin{tabular}{ll}
 Value     & Description                      \\
 linear    & linear scheme                    \\
 trilinear & uber next scheme                 \\
 \end{tabular}

 The linear temporal interpolation method computes the temporal weights
 based on two points. Ubernext computes weights based on three points.
 Currently the ubernext option is implemented only for the GSWP forcing.
 

 \begin{Verbatim}[frame=single]
Temporal interpolation method (met forcing): linear
 \end{Verbatim}

 
 \var{Enable new zterp correction (met forcing):}
 specifies whether to enable the new zterp correction.
 Acceptable values are:

 \begin{tabular}{ll}
 Value         & Description   \\
 \var{.false.} & do not enable \\
 \var{.true.}  & enable        \\
 \end{tabular}

 Defaults to .false..

 This is a scalar option, not per nest.

 This new zterp correction addresses an issue that occurs
 at sunrise/sunset for some forcing data-sets when running at small
 time-steps (like 15mn).  In these cases, swdown has a large unrealistic
 spike.  This correction removes the spike.  It also can affect swdown
 around sunrise/sunset by up 200 W/m2.  Users are advised to run their
 own tests and review swdown to determine which setting is best
 for them.

 For comparision against older LIS runs, set this option
 to \var{.false.}.
 

 \begin{Verbatim}[frame=single]
Enable new zterp correction (met forcing): .false.
 \end{Verbatim}

 
 \var{Temporal downscaling method:} specifies the
 temporal downscaling method to disaggregate a coarser forcing
 dataset into finer timesteps (e.g., go from daily to hourly).

 User-entry activated only when the ``Metforce temporal downscaling'' 
  run mode is selected.

 Acceptable values are:

 \begin{tabular}{ll}
 Value    & Description                                    \\
 ``Simple weighting'' & Use finer timescale forcing dataset \\
          & to estimate weights and downscale coarser forcing \\
          & dataset. The finer timescale forcing dataset should \\
          & defined first in the ldt.config file. \\
 \end{tabular}
 

 \begin{Verbatim}[frame=single]
Temporal downscaling method:    "Simple weighting"
 \end{Verbatim}

 
 \var{Processed metforcing output interval:} specifies the 
 output interval for the processed meteorological forcing files.
 Entries are character-based names, like 6hr or 1da.
 

 \begin{Verbatim}[frame=single]
Processed metforcing output interval:   "6hr"
 \end{Verbatim}

 
 \var{Metforcing processing interval:} specifies the 
 processing temporal interval for which meteorological forcing
 files are commonly and temporally aggregated (or downscaled) to
 when temporally downscaling a dataset.
 

 \begin{Verbatim}[frame=single]
Metforcing processing interval:     "1da"
 \end{Verbatim}

 
 \var{Statistical downscaling mode:} specifies the
  type of statistical downscaling method to be applied.

 User-entry activated only when the 
 ``Statistical downscaling of met forcing'' 
  run mode is selected.

 Acceptable values are:

 \begin{tabular}{ll}
 Value    & Description                                    \\
 ``downscale'' &  The downscale option is for bringing a coarser\\
          & climate model or reanalysis dataset to a finer scale \\ 
          & using statistical techniques (beyond interpolation).
 \end{tabular}
 

 \begin{Verbatim}[frame=single]
Statistical downscaling mode:       "downscale"
 \end{Verbatim}

 
 \var{Statistical downscaling method:} specifies the 
  method for downscaling or for climatology forcing generation.

 Current acceptable values are:

 \begin{tabular}{ll}
 Value    & Description                                    \\
 ``Climatology'' &  This option supports the generation of \\
  meteorological climatology files, for different forcing data. \\
 ``Bayesian merging'' &  \attention{specifies what?}
 \end{tabular}
 

 \begin{Verbatim}[frame=single]
Statistical downscaling method:     "Climatology"
 \end{Verbatim}

 
 \var{Forcing climatology temporal frequency of data:} specifies the 
 output time interval to which the forcing climatology will be 
 calculated on and written to.
 

 \begin{Verbatim}[frame=single]
Forcing climatology temporal frequency of data:   "6hr"
 \end{Verbatim}

 
 \var{Bayesian merging seasonal stratification type:} \attention{specifies what}
 

 \begin{Verbatim}[frame=single]
Bayesian merging seasonal stratification type:
 \end{Verbatim}

 
 \var{Forcing variables list file:} specifies the file containing
 the list of forcing variables to be used. Please refer to the 
 sample forcing\_variables.txt (Section~\ref{sec:forcingvars})
 file for a complete specification description. 
 

 \begin{Verbatim}[frame=single]
Forcing variables list file:     ./input/forcing_variables.txt
 \end{Verbatim}

 
 \var{LDT diagnostic file:} specifies the name of run time
 LDT diagnostic file.
 

 \begin{Verbatim}[frame=single]
LDT diagnostic file:           ldtlog
 \end{Verbatim}

 
 \var{Mask-parameter fill diagnostic file:} specifies the name 
 of the output diagnostic file for wherever mask-parameter
 points have forced agreement for a given landmask and parameter.
 

 \begin{Verbatim}[frame=single]
Mask-parameter fill diagnostic file:  LDTOUTPUT_temp/MPFilltest.log
 \end{Verbatim}

 
 \var{LDT output directory:} specifies the directory name for 
 outputs from LDT
 Acceptable values are any 40 character string.
 The default value is set to OUTPUT.
 

 \begin{Verbatim}[frame=single]
LDT output directory:       OUTPUT
 \end{Verbatim}

 
 \var{Undefined value:} specifies the undefined value.
 The default is set to -9999.
 

 \begin{Verbatim}[frame=single]
Undefined value:             -9999.0
 \end{Verbatim}

 
 \var{Number of ensembles per tile:} specifies the number of
 ensembles of tiles to be used. The value should be greater than
 or equal to 1.
 

 \begin{Verbatim}[frame=single]
Number of ensembles per tile:      1
 \end{Verbatim}

 
 The following options are used for subgrid tiling based on vegetation or other
  parameter types (e.g., soil type, elevation, etc.),
  and are required for generating an ensemble restart file or downscaling
  to a single-member restart file from an ensemble one. See the section
  on ensemble restart files for more information.

 \var{Maximum number of surface type tiles per grid:} defines the
 maximum surface type tiles per grid (this can be as many as the total
 number of vegetation types). 
 Note:
 Allowable values are greater than or equal to 1. Note that the entry
 here should exactly match the entry used in the lis.config file.
 

 \begin{Verbatim}[frame=single]
Maximum number of surface type tiles per grid: 1
 \end{Verbatim}

 
 \var{Minimum cutoff percentage (surface type tiles):} defines the
 smallest percentage of a cell for which to create a tile.
 The percentage value is expressed as a fraction.
 Allowable values are greater than or equal to 0. Note that the entry
 here should exactly match the entry used in the lis.config file. 
 

 \begin{Verbatim}[frame=single]
Minimum cutoff percentage (surface type tiles): 0.05
 \end{Verbatim}

 
 \var{Maximum number of soil texture tiles per grid:} defines the
 maximum soil texture tiles per grid (this can be as many as the total
 number of soil texture types). 
 Allowable values are greater than or equal to 1. Note that the entry
 here should exactly match the entry used in the lis.config file. 
 

 \begin{Verbatim}[frame=single]
Maximum number of soil texture tiles per grid: 1
 \end{Verbatim}

 
 \var{Minimum cutoff percentage (soil texture tiles):} defines the
 smallest percentage of a cell for which to create a tile.
 The percentage value is expressed as a fraction.
 Allowable values are greater than or equal to 0. Note that the entry
 here should exactly match the entry used in the lis.config file. 
 

 \begin{Verbatim}[frame=single]
Minimum cutoff percentage (soil texture tiles): 0.05
 \end{Verbatim}

 
 \var{Maximum number of soil fraction tiles per grid:} defines the
 maximum soil fraction tiles per grid (this can be as many as the total
 number of soil fraction types). 
 Allowable values are greater than or equal to 1. Note that the entry
 here should exactly match the entry used in the lis.config file. 
 

 \begin{Verbatim}[frame=single]
Maximum number of soil fraction tiles per grid: 1
 \end{Verbatim}

 
 \var{Minimum cutoff percentage (soil fraction tiles):} defines the
 smallest percentage of a cell for which to create a tile.
 The percentage value is expressed as a fraction.
 Allowable values are greater than or equal to 0. Note that the entry
 here should exactly match the entry used in the lis.config file. 
 

 \begin{Verbatim}[frame=single]
Minimum cutoff percentage (soil fraction tiles): 0.05
 \end{Verbatim}

 
 \var{Maximum number of elevation bands per grid:} defines the
 maximum elevation bands per grid (this can be as many as the total
 number of elevation band types). 
 Allowable values are greater than or equal to 1. Note that the entry
 here should exactly match the entry used in the lis.config file. 
 

 \begin{Verbatim}[frame=single]
Maximum number of elevation bands per grid: 1
 \end{Verbatim}

 
 \var{Minimum cutoff percentage (elevation bands):} defines the
 smallest percentage of a cell for which to create a tile.
 The percentage value is expressed as a fraction.
 Allowable values are greater than or equal to 0. Note that the entry
 here should exactly match the entry used in the lis.config file. 
 

 \begin{Verbatim}[frame=single]
Minimum cutoff percentage (elevation bands): 0.05
 \end{Verbatim}

 
 \var{Maximum number of slope bands per grid:} defines the
 maximum slope bands per grid (this can be as many as the total
 number of slope band types). 
 Allowable values are greater than or equal to 1. Note that the entry
 here should exactly match the entry used in the lis.config file. 
 

 \begin{Verbatim}[frame=single]
Maximum number of slope bands per grid: 1
 \end{Verbatim}

 
 \var{Minimum cutoff percentage (slope bands):} defines the
 smallest percentage of a cell for which to create a tile.
 The percentage value is expressed as a fraction.
 Allowable values are greater than or equal to 0. Note that the entry
 here should exactly match the entry used in the lis.config file. 
 

 \begin{Verbatim}[frame=single]
Minimum cutoff percentage (slope bands): 0.05
 \end{Verbatim}

 
 \var{Maximum number of aspect bands per grid:} defines the
 maximum aspect bands per grid (this can be as many as the total
 number of aspect band types). 
 Allowable values are greater than or equal to 1. Note that the entry
 here should exactly match the entry used in the lis.config file. 
 

 \begin{Verbatim}[frame=single]
Maximum number of aspect bands per grid: 1
 \end{Verbatim}

 
 \var{Minimum cutoff percentage (aspect bands):} defines the
 smallest percentage of a cell for which to create a tile.
 The percentage value is expressed as a fraction.
 Allowable values are greater than or equal to 0. Note that the entry
 here should exactly match the entry used in the lis.config file. 
 

 \begin{Verbatim}[frame=single]
Minimum cutoff percentage (aspect bands): 0.05
 \end{Verbatim}


 
 This section specifies the 2-d layout of the processors in a
 parallel processing environment. 

 \highlight{This is a new feature within LDT.}\\

 The user can specify the number of
 processors along the east-west dimension and north-south dimension
 using \var{Number of processors along x:} and 
 \var{Number of processors along y:}, respectively. Note that the layout 
 of processors should match the total number of processors used. For example, 
 if 8 processors are used, the layout can be specified as 1x8, 2x4, 4x2,
 or 8x1.
 

 \begin{Verbatim}[frame=single]
Number of processors along x:       2
Number of processors along y:       2
 \end{Verbatim}

 
 \var{Output methodology:} specifies whether to write output as a
 1-D array containing only land points or as a 2-D array containing
 both land and water points. 1-d tile space includes the subgrid
 tiles and ensembles. 1-d grid space includes a vectorized, land-only
 grid-averaged set of values.
 Acceptable values are:

 \begin{tabular}{ll}
 Value          & Description                         \\
 ``none''         & Do not write output               \\
 ``1d tilespace'' & Write output in a 1-D tile domain \\
 ``2d gridspace'' & Write output in a 2-D grid domain \\
 ``1d gridspace'' & Write output in a 1-D grid domain \\
 \end{tabular}

 The default value is "2d gridspace".
 

 \begin{Verbatim}[frame=single]
Output methodology: "2d gridspace"
 \end{Verbatim}

 
 \var{Output data format:} specifies the format of the model output data.
 Acceptable values are:

 \begin{tabular}{ll}
 Value      & Description                       \\
 ``binary'' & Write output in binary format     \\
 ``grib1''  & Write output in GRIB-1 format     \\
 ``netcdf'' & Write output in netCDF format     \\
 \end{tabular}

 The default value is "netcdf".
 

 \begin{Verbatim}[frame=single]
Output data format: netcdf
 \end{Verbatim}

 
 \var{Output naming style:} specifies the style of the model output
 names and their organization.
 Acceptable values are:

 \begin{tabular}{ll}
 Value                 & Description                       \\
 ``2 level hierarchy'' & 2 levels of hierarchy             \\
 ``3 level hierarchy'' & 3 levels of hierarchy             \\
 ``4 level hierarchy'' & 4 levels of hierarchy             \\
 ``WMO convention''    & WMO convention for weather codes  \\
 \end{tabular}

 The default value is "3 level hierarchy".
 

 \begin{Verbatim}[frame=single]
Output naming style: "3 level hierarchy"
 \end{Verbatim}

 
 \subsection{Domain specification} \label{ssec:domainspec}
 This section of the config file specifies the LIS running domain
 (domain over which the simulation is carried out).
 The specification of the LIS Run domain section depends on the type of
 LIS domain and projection used.
 Section~\ref{ssec:driveropts} lists the projections that LIS supports.
 

 
 \var{Map projection of the LIS domain:} specifies the output 
 LIS domain grid to be used with LIS.
 Acceptable values are:

 \begin{tabular}{ll}
 Value    & Description                                              \\
 latlon   & Lat/Lon projection with SW to NE data ordering           \\
 lambert  & Lambert conformal projection with SW to NE data ordering \\
 gaussian & Gaussian domain                                          \\
 polar    & Polar stereographic projection with SW to NE data
            ordering                                                 \\
 hrap     & HRAP domain (based on polar stereographic projection)    \\
 mercator & Mercator projection with SW to NE data ordering          \\
 \end{tabular}
 

 \begin{Verbatim}[frame=single]
Map projection of the LIS domain:      latlon
 \end{Verbatim}

 
 \subsubsection{Cylindrical lat/lon} \label{sssec:run_latlon}
 This section describes how to specify a cylindrical latitude/longitude
 projection.
 See Appendix~\ref{sec:d_latlon_example} for more details about
 setting these values.
 

 \begin{Verbatim}[frame=single]
Run domain lower left lat:           25.625
Run domain lower left lon:         -124.125
Run domain upper right lat:          52.875
Run domain upper right lon:         -67.875
Run domain resolution (dx):           0.25
Run domain resolution (dy):           0.25
 \end{Verbatim}

 
 \subsubsection{Lambert conformal} \label{sssec:run_lambert}
 This section describes how to specify a Lambert conformal
 projection.
 See Appendix~\ref{sec:d_lambert_example} for more details about
 setting these values.
 

 \begin{Verbatim}[frame=single]
Run domain lower left lat:           32.875
Run domain lower left lon:         -104.875
Run domain true lat1:                36.875
Run domain true lat2:                36.875
Run domain standard lon:            -98.875
Run domain resolution:               25.0
Run domain x-dimension size:          40
Run domain y-dimension size:          30
 \end{Verbatim}

 
 \subsubsection{Gaussian} \label{sssec:run_gaussian}
 This section describes how to specify a Gaussian
 projection.
 See Appendix~\ref{sec:d_gaussian_example} for more details about
 setting these values.
 

 \begin{Verbatim}[frame=single]
Run domain first grid point lat:     -89.27665
Run domain first grid point lon:       0.0
Run domain last grid point lat:       89.27665
Run domain last grid point lon:      -0.9375
Run domain resolution dlon:           0.9375
Run domain number of lat circles:       95
 \end{Verbatim}

 
 \subsubsection{Polar stereographic} \label{sssec:run_ps}
 This section describes how to specify a polar stereographic
 projection.
 See Appendix~\ref{sec:d_ps_example} for more details about
 setting these values.
 

 \begin{Verbatim}[frame=single]
Run domain lower left lat:           32.875
Run domain lower left lon:         -104.875
Run domain true lat:                 36.875
Run domain standard lon:            -98.875
Run domain orientation:               0.0
Run domain resolution:               25.0
Run domain x-dimension size:          40
Run domain y-dimension size:          30
 \end{Verbatim}

 
 \subsubsection{HRAP} \label{sssec:run_hrap}
 This section describes how to specify a HRAP 
 projection.
 See Appendix~\ref{sec:d_hrap_example} for more details about
 setting these values.

 

 \begin{Verbatim}[frame=single]
Run domain lower left hrap y:          48
Run domain lower left hrap x:          17
Run domain hrap resolution:            1
Run domain x-dimension size:          1043
Run domain y-dimension size:          774
 \end{Verbatim}

 
 \subsubsection{Mercator} \label{sssec:run_mercator}
 This section describes how to specify a Mercator
 projection.
 See Appendix~\ref{sec:d_mercator_example} for more details about
 setting these values.
 

 \begin{Verbatim}[frame=single]
Run domain lower left lat:          32.875
Run domain lower left lon:        -104.875
Run domain true lat1:               36.875
Run domain standard lon:           -98.875
Run domain resolution:              25.0
Run domain x-dimension size:         40
Run domain y-dimension size:         30
 \end{Verbatim}



 
 \subsection{Parameters} \label{ssec:parameters}
 

 
 \var{Landcover data source:} specifies the land cover 
 dataset source to be read in.
 Current landcover source options include:

 \begin{tabular}{ll}
 Value         & Description               \\
 AVHRR         & Univ. of Maryland 1992-93 AVHRR landcover map.  \\
               &   Please see: http://glcf.umd.edu/data/landcover/ \\ 
 AVHRR\_GFS    & Similar to ``AVHRR'' option above but on a GFS grid. \\
 MODIS\_Native & Terra-MODIS sensor-based IGBP land classification map, \\
               &   modified by NCEP. For more info, please see:  \\
               &   http://www.ral.ucar.edu/research/land/technology/noahmp\_lsm.php \\
 MODIS\_LIS    & Similar dataset as ``MODIS\_Native''  \\
               &   above but processed by LIS-team.   \\
 USGS\_Native  & The 24-category USGS native landcover map. See:  \\
               &   http://www.ral.ucar.edu/research/land/technology/noahmp\_lsm.php  \\
 USGS\_LIS     & Similar dataset as ``USGS\_Native'' but processed by LIS-team.    \\
 ALMIPII       & AMMA/ALMIP Phase-2 landcover input option.  For more info:    \\ 
               &   http://www.cnrm.meteo.fr/amma-moana/amma\_surf/almip2/input.html \\
 CLSMF2.5      & CLSM Fortuna 2.5 landcover dataset.  \\
 VIC412        & Variable Infiltration Capacity model, v4.1.2, UMD land cover.  \\
 ISA           & Impervious Surface Area (ISA) landcover dataset.  \\
 CONSTANT      & Ability to set a constant landcover type for a set classification.  \\
 \end{tabular}
 

 \begin{Verbatim}[frame=single]
Landcover data source:    "MODIS_Native"
 \end{Verbatim}

 
 \var{Landcover classification:} specifies the land cover 
 classification type.  Land cover or use classification types
 have been developed over the years by various organizations
 (e.g., USGS, IGBP) and research groups (e.g., satellite missions 
 associated with groups, like UMD, BU, etc.).
 For more information on some of these different land classifications
 and their characteristics, please refer to:
  http://edc2.usgs.gov/glcc/globdoc2\_0.php

 Acceptable values are:

 \begin{tabular}{ll}
 Value     & Description               \\
 UMD       &  14 Landcover types       \\
 IGBP      &  16 Landcover types       \\
 USGS      &  24 Landcover types       \\
 IGBPNCEP  &  20 Landcover types       \\
 MOSAIC    &   7 Landcover types       \\
 ISA       &  13 Landcover types       \\
 CONSTANT  &   2 Landcover types       \\
 \end{tabular}
 

 \begin{Verbatim}[frame=single]
Landcover classification:     "UMD"  
 \end{Verbatim}

 
 \var{Landcover file:} specifies the location of the vegetation
 classification file.

 \var{Landcover map projection:} specifies the projection of the
 landcover map data.

 \var{Landcover spatial transform:} indicates which spatial transform
 (i.e., upscale or downscale) type is to be applied to the landcover
 map.  Options include:

 \begin{tabular}{ll}
 Value & Description                                          \\
 none  & Data is on same grid as LIS output domain            \\
 mode  & Upscale by selecting dominant type for each gridcell \\
 neighbor &  Use nearest neighbor to select closest value  \\
 tile  & Read in tiled data or upscale by estimating gridcell fractions\\
 \end{tabular}
 Note: ``tile'' is a special case for landcover, which allows
 for reading in landcover data already in tiled form, or creating
 tiles from finer resolution landcover data.
 

 \begin{Verbatim}[frame=single]
Landcover file:               ../input/1KM/landcover_UMD.1gd4r
Landcover spatial transform:     tile
 \end{Verbatim}

 
 \var{Landcover fill option:} specifies the landcover classification
  data fill option.  Options include:

 \begin{tabular}{ll}
 Value      & Description                                  \\
 none       &  Do not apply missing value fill routines    \\
 neighbor   &  Use nearest neighbor to fill missing value  \\
 \end{tabular}

 \var{Landcover fill value:} indicates which landcover
 value to be used if an arbitrary value fill is needed. 
 (For example, when the landmask indicates a land point but no existing 
 landcover point, a value of 5 could be assigned if 
 no nearest neighbor values exists to fill).

 \var{Landcover fill radius:} specifies the radius with which
 to search for nearby value(s) to help fill the missing value.
 

 \begin{Verbatim}[frame=single]
Landcover fill option:   neighbor    # none, neighbor
Landcover fill radius:     3.        # Number of pixels to search for neighbor
Landcover fill value:      5.        # Static value to fill where missing
 \end{Verbatim}

 
 This section also outlines the domain specifications of the 
 landcover (and for now landmask) data. 
 If the map projection of parameter data is specified to be lat/lon, 
 the following configuration should be used for specifying landcover
 data.
 Note: The Landcover grid domain inputs below are really only required
  for the ``\_LIS'' data source options and that do not include ``\_Native''
  in the data source entries. All native parameters do not require
  the below inputs for LDT.
 See Appendix~\ref{sec:d_latlon_example} for more details about
 setting these values. 
 

 \begin{Verbatim}[frame=single]
Landcover map projection:        latlon
Landcover lower left lat:       -59.995
Landcover lower left lon:      -179.995
Landcover upper right lat:       89.995
Landcover upper right lon:      179.995
Landcover resolution (dx):        0.01
Landcover resolution (dy):        0.01
 \end{Verbatim}

 
 \var{Create or readin landmask:} offers the user the option
 to create or read in land/water mask file information.
 Options include the ability to impose the mask on landcover
 and also the other parameter fields.
 

 \begin{Verbatim}[frame=single]
Create or readin landmask:      "readin"
 \end{Verbatim}

 
 \var{Landmask data source:} specifies the land mask
 dataset source to be read in.  If the user is interested in
 only using the selected landcover data source, then the user 
 can select the same option for the landmask data source.
  
 Other current landmask source options include:

 \begin{tabular}{ll}
 Value     & Description                                           \\
 MOD44W    & The MODIS 44W land-water mask was developed and provided by: \\
           &  http://glcf.umd.edu/data/watermask/ \\
 HYMAP     & The HYMAP basin area mask option. \\
 \end{tabular}
 

 \begin{Verbatim}[frame=single]
Landmask data source:   "MODIS_Native"
 \end{Verbatim}

 
 \var{Landmask file:} specifies the location of land/water mask file. \\
 Note: If reading in the MOD44W land-water mask, make sure to enter   
 'MOD44W' Landmask data source entry.
 

 \begin{Verbatim}[frame=single]
Landmask file:        ../input/1KM/landmask_UMD.1gd4r
 \end{Verbatim}

 
 \var{Landmask spatial transform:} indicates which spatial transform
 (i.e., upscale or downscale) type is to be applied to the landmask
 map.  Options include:

 \begin{tabular}{ll}
 Value     & Description                                           \\
 none      &  Data is on same grid as LIS output domain            \\
 mode      &  Upscale by selecting dominant type for each gridcell \\
 neighbor  &  Use nearest neighbor when downscaling (or upscaling, if needed) \\
 \end{tabular}
 

 \begin{Verbatim}[frame=single]
Landmask spatial transform:    none     
 \end{Verbatim}

 
 \var{Landmask map projection:} specifies the projection of the
 landmask map data.
 

 \begin{Verbatim}[frame=single]
Landmask map projection:       latlon
 \end{Verbatim}

 
 This section also outlines the domain specifications of the
 land water/mask data.  The landmask map projection and extents
 are only needed if you specify ``readin'' for mask type and
 if the landmask data source is ``MOD44W'' or ``\_LIS''.  
 
 If the map projection of parameter data is specified to be ``latlon'',
 the following extents and resolution configuration should be used 
 for specifying landmask data.
 See Appendix~\ref{sec:d_latlon_example} for more details about
 setting these values.

 Future landmask data sets will have the projection, grid extents and 
 resolution on the data reader side and not needed to be specified in
 the ldt config file, depending on the data source.
 

 \begin{Verbatim}[frame=single]
Landmask map projection:        latlon
Landmask lower left lat:       -59.995
Landmask lower left lon:      -179.995
Landmask upper right lat:       89.995
Landmask upper right lon:      179.995
Landmask resolution (dx):        0.01
Landmask resolution (dy):        0.01
 \end{Verbatim}



 
 \var{Regional mask data source:} specifies a regional land mask
 dataset source to be read in.  Should either match grid domain
 or be smaller to the LIS run domain.
  
 \begin{tabular}{ll}
 Value     & Description                  \\
 ESRI      & Binary file type mask produced in ESRI-GIS software. \\
 WRSI      & A BIL-format (binary) mask file associated with WRSI model. \\
 \end{tabular}
 

 \begin{Verbatim}[frame=single]
Regional mask data source:   "none"
 \end{Verbatim}

 
 \var{Regional mask file:} specifies the location of a regional
 mask file.
 This file can be either an index-based state, country, basin,
 catchment, etc.\ map used to mask further beyond the main
 water/land mask.
 

 \begin{Verbatim}[frame=single]
Regional mask file:    ../input/1KM/regional_statemask.1gd4r
 \end{Verbatim}

 
 \var{Regional mask map projection:} specifies the projection of the
 regional mask albedo map data.
 

 \begin{Verbatim}[frame=single]
Regional mask map projection:
 \end{Verbatim}

 
 \var{Clip landmask with regional mask:} A logical-based option 
  that uses the regional mask to `clip' the original landmask that is
  read-in or created.
  .true. turns on the "clipping" option.
 

 \begin{Verbatim}[frame=single]
Clip landmask with regional mask:  .true.
 \end{Verbatim}

 
 \var{Regional mask spatial transform:} indicates which spatial
 transform (i.e., upscale or downscale) type is to be applied
 to a regional mask map.  Options include:

 \begin{tabular}{ll}
 Value     & Description                                          \\
 none      &  Data is on same grid as LIS output domain           \\
 neighbor  &  Use nearest neighbor to select closest value        \\
 mode      &  Upscale by selecting dominant type for each gridcell \\
 \end{tabular}
 

 \begin{Verbatim}[frame=single]
Regional mask spatial transform:   mode
 \end{Verbatim}

 
 This section also outlines the domain specifications of the
 regional-based land mask data.
 If the map projection of parameter data is specified to be lat/lon,
 the following configuration should be used for specifying regional
 mask data.
 
 See Appendix~\ref{sec:d_latlon_example} for more details about
 setting these values.
 

 \begin{Verbatim}[frame=single]
Regional mask map projection:        latlon
Regional mask lower left lat:       -59.995
Regional mask lower left lon:      -179.995
Regional mask upper right lat:       89.995
Regional mask upper right lon:      179.995
Regional mask resolution (dx):        0.01
Regional mask resolution (dy):        0.01
 \end{Verbatim}

 
 \var{Rootdepth data source:} specifies the source
  of the vegetation root depth dataset.
  Options include:

 \begin{tabular}{ll}
 Value   & Description                         \\
 none    & No data         \\
 ALMIPII  & ALMIP II root depth field \\
 \end{tabular}
 

 \begin{Verbatim}[frame=single]
Rootdepth data source:    none
 \end{Verbatim}

 
 \var{Root depth file:} specifies the path
  and name of the root depth file.
  Options include:

 \begin{tabular}{ll}
 Value   & Description                         \\
 none    & No data         \\
 ALMIPII  & ALMIP II root depth field \\
 \end{tabular}
 

 \begin{Verbatim}[frame=single]
Root depth file:      none
 \end{Verbatim}


 
 \subsection{Crop-Irrigation Parameters} \label{ssec:cropirrigparams}
 


 
 \var{Incorporate crop information:} specifies the logical flag
 with which to turn on the inclusion of crop information and also 
 to allow the user to enter additional options that can ensure
 crop, landcover, and irrigation features are agreement.
 

 \begin{Verbatim}[frame=single]
Incorporate crop information:   .false.
 \end{Verbatim}

 
 \var{Crop type data source:} specifies the crop type map
 dataset source to be read in.
 Current landcover source options include:

 \begin{tabular}{ll}
 Value         & Description               \\
 UMDCROPMAP    & UMD+CROPMAP crop type map.  For more info,  \\
               &   please refer to Ozdogan et al. (2010; JHM).  \\
 Monfreda08    & FAOSTAT05 crop type maps.  For more info,  \\
               &   please refer to Monfreda et al. (2008).  \\
 CONSTANT      & Ability to set a constant crop type for a set classification. \\
 \end{tabular}
 

 \begin{Verbatim}[frame=single]
Crop type data source:  "none" 
 \end{Verbatim}

 
 \var{Crop classification:} specifies the crop classification
 system used to determine the range of crops indexed for a particular
 crop library source.

 \begin{tabular}{ll}
 Value     & Description                                          \\
 none      &  Data is on same grid as LIS output domain           \\
 CROPMAP   &   19 classes; named by Ozdogan et al.(2010), used Leff et al.(2004) \\
 FAOSTAT01 &   19 classes; Used by Leff et al.(2004), considered obsolete \\
 FAOSTAT05 &  175 classes; Used by Monfreda et al. (2008) \\
 \end{tabular}
 

 \begin{Verbatim}[frame=single]
Crop classification:       "FAOSTAT01"  
 \end{Verbatim}

 \var{Crop library directory:} specifies the source of the
 crop library and inventory of crop classes, related to the 
 ``Crop classification:'' entry (see above).

 \begin{Verbatim}[frame=single]
Crop library directory:  "../LS_PARAMETERS/crop_params/Crop.Library.Files/"
 \end{Verbatim}

 \var{Assign crop value type:} specifies the type of crop presence,
 such as a ``single'' crop or ``multiple'' crops given within a 
 gridcell.  Currently, only the ``single'' option is supported.

 \begin{Verbatim}[frame=single]
Assign crop value type:    "none"
 \end{Verbatim}

 
 \var{Assign single crop value:} specifies whether to assign a single
  crop value from an actual crop library inventory, such as FAOSTAT01,
  which is also known as the CROPMAP classification used in Ozdogan et al. (2010).
  By turning on this option (.true.), you can they specify what type of
  crop you want to assign, like ``maize'' to the user-specified option,
  \var{Default crop type:}.  If ``maize'' was entered, then wherever the landcover
  map indicated a generic ``cropland'', the crop type field would be given a
  dominant ``maize'' type.

 \begin{tabular}{ll}
 Value     & Description                                           \\
  .false.  &  Do not assign a single crop class to the crop type field. \\
  .true.   &  Assign a single crop type, like ``maize'' to the crop type field. \\
 \end{tabular}
 

 \begin{Verbatim}[frame=single]
Assign single crop value:     .true.    
 \end{Verbatim}

 
 \var{Default crop type:} specifies the default crop type that the
  user can enter and can be used in conjunction with assigning a single
  crop type value (see above).
 

 \begin{Verbatim}[frame=single]
Default crop type:           "maize"   
 \end{Verbatim}

 
 \var{Crop type file:} specifies the location of a crop type file.
 This file contains different crop types that can be used in 
 in conjunction with a selected land cover map (as above).
 

 \begin{Verbatim}[frame=single]
Crop type file:  ./irrigation/conus_modis/UMD_N125C19.1gd4r
 \end{Verbatim}

 
 \var{Crop map spatial transform:} indicates which spatial transform
 (i.e., upscale or downscale) type is to be applied to a crop type
 map.  Options include:

 \begin{tabular}{ll}
 Value   & Description                                          \\
 none    & Data is on same grid as LIS output domain            \\
 mode    & Upscale by selecting dominant type for each gridcell \\
 tile    & Read in tiled data or upscale by estimating gridcell fractions \\
 \end{tabular}

 \textbf{NOTE:} LIS-7 will be expecting `mode' or dominant crop type per gridcell
    at this time. Future versions will include landcover-crop tile options.

 

 \begin{Verbatim}[frame=single]
Crop map spatial transform:   mode
 \end{Verbatim}


 
 \var{Irrigation type data source:} specifies the irrigation method type 
 dataset source to be read in.
 Current source options include:

 \begin{tabular}{ll}
 Value       & Description               \\
 GRIPC       & Irrigation map, by Salmon (2013).  \\
 \end{tabular}
 

 \begin{Verbatim}[frame=single]
Irrigation type data source:  "none"
 \end{Verbatim}

 
 \var{Irrigation type map:} specifies the location of an irrigation
 type file.
 This file contains different irrigation categories (types) that
 can be used in conjunction with an irrigation fraction map.
 
 A special land-use/irrigation-related map, known as the
 Global Rain-Fed, Irrigated, and Paddy Croplands (GRIPC) Dataset
 (Salmon, 2013), has also been implemented as an option to
 LDT.  Currently, no models in LIS utilize this map but 
 opportunities exist for the user community to utilize for their
 landcover and irrigation modeling needs.
 

 \begin{Verbatim}[frame=single]
Irrigation type map: ../LS_PARAMETERS/irrigation/irrigtype_map.bin
 \end{Verbatim}

 
 \var{Irrigation type spatial transform:} indicates which spatial transform
 (i.e., upscale or downscale) type is to be applied to
 irrigation-related maps.  Options include:

 \begin{tabular}{ll}
 Value   & Description                                   \\
 none    & Data is on same grid as LIS output domain            \\
 mode    & Upscale by selecting dominant type for each gridcell \\
 neighbor &  Use nearest neighbor to select closest value  \\
 tile    & Read in tiled data or upscale by estimating gridcell fractions\\
 \end{tabular}
 

 \begin{Verbatim}[frame=single]
Irrigation type spatial transform:    mode
 \end{Verbatim}

 
 \var{Irrigation fraction data source:} specifies the irrigation method type 
 dataset source to be read in.
 Current source options include:

 \begin{tabular}{ll}
 Value       & Description               \\
 MODIS\_OG   & Irrigation area fraction map by Ozdogan and Gutman (2008) \\
 GRIPC       & Irrigation area fraction map by Salmon (2013) \\
 \end{tabular}
 

 \begin{Verbatim}[frame=single]
Irrigation fraction data source:  "none"
 \end{Verbatim}

 
 \var{Irrigation fraction map:} specifies the location of an
 irrigation fraction map file.
 This file contains irrigation fraction (gridcell-based) that
 can be used in conjunction with an irrigation type map.
 

 \begin{Verbatim}[frame=single]
Irrigation fraction map:  ../irrigation/irrig.percent.eighth.1gd4r
 \end{Verbatim}

 
 \var{Irrigation fraction spatial transform:} indicates which spatial transform
 (i.e., upscale or downscale) type is to be applied to
 irrigation-related maps.  Options include:

 \begin{tabular}{ll}
 Value   & Description                                   \\
 none    & Data is on same grid as LIS output domain     \\
 average & Upscale by averaging values for each gridcell \\
 \end{tabular}
 

 \begin{Verbatim}[frame=single]
Irrigation fraction spatial transform:     none
 \end{Verbatim}


 
 \subsection{Soil Parameters} \label{ssec:soilspecparams}
 

 
 Soils maps

 \var{Sand fraction map:} specifies the sand fraction map file.

 \var{Clay fraction map:} specifies the clay fraction map file.

 \var{Silt fraction map:} specifies the silt map file.


 \var{Porosity data source:} specifies the soil porosity
 dataset source to be read in. Current source options include:

 \begin{tabular}{ll}
 Value      & Description    \\
 FAO          &  LIS-team produced soil porosity data source. \\
 CLSMF2.5     &  Similar to the above option but for CLSM F2.5 model. \\
 CONSTANT     &  User can select a constant value. \\
 \end{tabular}

 \var{Porosity map:} specifies porosity map file.

 \var{Soildepth data source:} specifies the soildepth dataset source
 to be read in.  Current source option is:

 \begin{tabular}{ll}
 Value      & Description    \\
   ALMIPII  &  ALMIPII soil depth data source. \\
 \end{tabular}

 \var{Soil depth map:} specifies the soil depth map file.

 \var{Saturated matric potential map:} specifies saturated matric
 potential map file.

 \var{Saturated hydraulic conductivity map:} specifies saturated
 hydraulic conductivity map file.

 \var{b parameter map:} specifies b parameter map file.

 

 \begin{Verbatim}[frame=single]
Sand fraction map:        ../input/25KM/sand_FAO.1gd4r
Clay fraction map:        ../input/25KM/clay_FAO.1gd4r
Silt fraction map:        ../input/25KM/silt_FAO.1gd4r
Porosity data source:        none
Porosity map:                  
Saturated matric potential map:       
Saturated hydraulic conductivity map: 
b parameter map:                      
 \end{Verbatim}

 
 \var{Soil fraction data source:} specifies the source
  of the soil fraction dataset.
  Options include:

 \begin{tabular}{ll}
 Value   & Description                         \\
 none    & No soil fraction data source        \\
 FAO     & FAO soil fraction percentage fields \\
 STATSGO\_LIS & LIS-team derived STATSGO v1 soil fraction fields \\
 ALMIPII  & ALMIP II soil fraction percentage fields \\
 CONSTANT & If user wants to set a constant soil fraction values \\
 \end{tabular}
 

 \begin{Verbatim}[frame=single]
Soil fraction data source:     FAO
 \end{Verbatim}

 
 \var{Soil fraction number of bands:} specifies the number of
 soil fraction bins to turn on soil fraction tiling capability.

 

 \begin{Verbatim}[frame=single]
Soil fraction number of bands:     1
 \end{Verbatim}

 
 \var{Soils spatial transform:} indicates which spatial transform
 (i.e., upscale or downscale) type is to be applied to the soils
 maps.  Options include:

 \begin{tabular}{ll}
 Value   & Description                                           \\
 none      & Data is on same grid as LIS output domain           \\
 average   & Upscale by averaging values for each gridcell       \\
 neighbor  & Reinterpolate by selecting nearest gridcell neighbor \\
 bilinear  & Reinterpolate by using bilinear interpolation        \\
 budget-bilinear & Reinterpolate by using conservative, budget-bilinear \\
 tile      &  Read in tiled data or upscale by estimating gridcell
              fractions                                            \\
 \end{tabular}
 

 \begin{Verbatim}[frame=single]
Soils spatial transform:     average 
 \end{Verbatim}

 
 \var{Soils map projection:} specifies the projection of the
 soils map data.

 \var{Soils fill option:} specifies the general soil
 data (e.g., fractions) fill option.  Options include:

 \begin{tabular}{ll}
 Value    & Description                                 \\
 none     &  Do not apply missing value fill routines   \\
 neighbor &  Use nearest neighbor to fill missing value \\
 \end{tabular}

 By selecting the soils fill option, neighbor, this activates the
 need to enter values for the Soils fill radius and fill value, as
 shown below.  If a porosity map is read in and the soils fill option
 is set to neighbor, the user can then enter a fill value for porosity
 to ensure mask-parameter agreement. 

 \var{Soils fill radius:} specifies the radius with which
 to search for nearby value(s) to help fill the missing value.

 \var{Soils fill value:} indicates which soils
 value to be used if an arbitrary value fill is needed. 
 (For example, when the landmask indicates a land point but no existing 
 soils value, a value of 0.33 could be assigned if 
 no nearest neighbor values exists to fill).

 \var{Porosity fill value:} indicates which porosity
 value to be used if an arbitrary value fill is needed. 
 (For example, when the landmask indicates a land point but no existing 
 porosity value, a value of 0.30 could be assigned if 
 no nearest neighbor values exists to fill).
 

 \begin{Verbatim}[frame=single]
Soils fill option:   neighbor 
Soils fill radius:   3
Soils fill value:    0.33
Porosity fill value: 0.30
 \end{Verbatim}

 
 If the map projection of parameter data is specified to be lat/lon,
 the following configuration should be used for specifying soils data,
 if the data source option has a ``\_LIS'' in the name.
 See Appendix~\ref{sec:d_latlon_example} for more details about
 setting these values.
 

 \begin{Verbatim}[frame=single]
Soils map projection:        latlon
Soils lower left lat:      -59.87500
Soils lower left lon:     -179.87500
Soils upper right lat:      89.87500
Soils upper right lon:     179.87500
Soils resolution (dx):       0.2500
Soils resolution (dy):       0.2500
 \end{Verbatim}

 
 \var{Hydrologic soil group source:} specifies the 
 hydrological soil group (HSG) data source.
 Options include:

 \begin{tabular}{ll}
 Value   & Description                         \\
 none      & No HSG data source                \\
 STATSGOv1 & STATSGO v1 HSG data source        \\
 \end{tabular}
 

 \begin{Verbatim}[frame=single]
Hydrologic soil group source:        STATSGOv1
 \end{Verbatim}

 
 \var{Hydrologic soil group map:} specifies the 
 path and filename for the HSG input file.
 

 \begin{Verbatim}[frame=single]
Hydrologic soil group map:  ./input/STATSGO_v1/hsgpct.bsq
 \end{Verbatim}

 
 \var{Bulk density data source:} specifies the
 source of the soil bulk density data type.
 Currently no options supported at this time.
 

 \begin{Verbatim}[frame=single]
Bulk density data source:     none
 \end{Verbatim}

 
 \var{Water capacity data source:} specifies the
 source of the water holding capacity data type.
 Currently no options supported at this time.
 

 \begin{Verbatim}[frame=single]
Water capacity data source:   none
 \end{Verbatim}

 
 \var{Rock volume data source:} specifies the
 source of the amount of rock volume data type.
 Currently no options supported at this time.
 

 \begin{Verbatim}[frame=single]
Rock volume data source:   none
 \end{Verbatim}

 
 \var{Rock frag class data source:} specifies the
 source of the rock fragment classification type.
 Currently no options supported at this time.
 

 \begin{Verbatim}[frame=single]
Rock frag class data source:  none
 \end{Verbatim}

 
 \var{Permeability data source:} specifies the
 source of the permeability data type.
 Currently no options supported at this time.
 

 \begin{Verbatim}[frame=single]
Permeability data source:   none
 \end{Verbatim}


 
 \var{Soil texture data source:} specifies the soil texture 
 dataset source to be read in.  
  
 Current soil texture source options include:

 \begin{tabular}{ll}
 Value               & Description             \\
 STATSGOFAO\_Native  & The blended STATSGOv1 and FAO soil texture map. See: \\
                     & http://www.ral.ucar.edu/research/land/technology/lsm.php \\
 STATSGOFAO\_LIS     & Similar dataset as to the above one but processed by LIS-team.\\
 FAO                 & FAO or Reynolds et al. (1999) soil texture map. \\
 ZOBLER\_GFS         & Similar to above but on a GFS run domain.  \\
 STATSGOv1           & The STATSGOv1-only soil texture map.  \\
 CONSTANT            & User can set a constant soil texture class. \\
 \end{tabular}
 

 \begin{Verbatim}[frame=single]
Soil texture data source:   "STATSGOFAO_Native"
 \end{Verbatim}

 
 \var{Soil texture map:} specifies the soil texture file.

 \var{Soil texture spatial transform:} indicates which spatial transform
 (i.e., upscale or downscale) type is to be applied to the soil texture
 map.  Options include:

 \begin{tabular}{ll}
 Value & Description                                          \\
 none  & Data is on same grid as LIS output domain            \\
 mode  & Upscale by selecting dominant type for each gridcell \\
 neighbor & Upscale by using nearest valid value for each gridcell \\
 tile  & Read in tiled data or upscale by estimating gridcell
         fractions                                            \\
 \end{tabular}
 

 \begin{Verbatim}[frame=single]
Soil texture map:  ../input/25KM/soiltexture_STATSGO-FAO.1gd4r 
Soil texture spatial transform:     none
 \end{Verbatim}

 
 \var{Soil texture map projection:} specifies the projection of the
 soil texture map data.

 \var{Soil texture fill option:} specifies the soil texture
  data fill option.  Options include:

 \begin{tabular}{ll}
 Value    & Description                                \\
 none     & Do not apply missing value fill routines   \\
 neighbor & Use nearest neighbor to fill missing value \\
 \end{tabular}

 \var{Soil texture fill value:} indicates which soil texture 
 value to be used if an arbitrary value fill is needed. 
 (For example, when the landmask indicates a land point but no existing 
 soil texture value, a value of 6 could be assigned if 
 no nearest neighbor values exists to fill).

 \var{Soil texture fill radius:} specifies the radius with which
 to search for nearby value(s) to help fill in the missing value.
 

 \begin{Verbatim}[frame=single]
Soil texture fill option:      neighbor
Soil texture fill radius:         3.
Soil texture fill value:          6.
 \end{Verbatim}

 
 If the map projection of parameter data is specified to be lat/lon,
 the following configuration should be used for specifying soil
 texture data, if the data source option has a ``\_LIS'' in the name.
 See Appendix~\ref{sec:d_latlon_example} for more details about
 setting these values.
 

 \begin{Verbatim}[frame=single]
Soil texture map projection:        latlon
Soil texture lower left lat:       -59.87500
Soil texture lower left lon:      -179.87500
Soil texture upper right lat:       89.87500
Soil texture upper right lon:      179.87500
Soil texture resolution (dx):        0.2500
Soil texture resolution (dy):        0.2500
 \end{Verbatim}

 
 \var{Soil color map projection:} specifies the projection of the
 soil color map data.

 \var{Soil color data source:} specifies the soil color data source.
  Current option is:  FAO

 \var{Soil color map:} specifies the soil color map file.
  This soil map is mainly used by the Community Land Model (version 2).

 \var{Soil color spatial transform:} indicates which spatial transform
 (i.e., upscale or downscale) type is to be applied to the soil color
 map.  Options include:

 \begin{tabular}{ll}
 Value & Description                                          \\
 none  & Data is on same grid as LIS output domain            \\
 mode  & Upscale by selecting dominant type for each gridcell \\
 neighbor  & Reinterpolate by selecting nearest gridcell neighbor \\
 \end{tabular}
 

 \begin{Verbatim}[frame=single]
Soil color data source:        none
Soil color map:         
Soil color spatial transform:     none
 \end{Verbatim}

 
 If the map projection of parameter data is specified to be lat/lon,
 the following configuration should be used for specifying soil color
 data, data source option ``FAO'' or has a ``\_LIS'' in the name. 
 See Appendix~\ref{sec:d_latlon_example} for more details about
 setting these values.
 

 \begin{Verbatim}[frame=single]
Soil color map projection:         latlon
Soil color lower left lat:       -59.87500
Soil color lower left lon:      -179.87500
Soil color upper right lat:       89.87500
Soil color upper right lon:      179.87500
Soil color resolution (dx):        0.2500
Soil color resolution (dy):        0.2500
 \end{Verbatim}


 
 \subsection{Topography Parameters} \label{ssec:topoparams}
 

 
 \var{Elevation data source:} specifies the elevation
 dataset source to be read in.  

 \var{Slope data source:} specifies the slope
 dataset source to be read in.  

 \var{Aspect data source:} specifies the aspect
 dataset source to be read in.  

 \var{Curvature data source:} specifies the curvature
 dataset source to be read in.  

 Current options include:

 \begin{tabular}{ll}
 Value            & Description             \\
 GTOPO30\_Native  & The GTOPO30 elevation map native source. See: \\
                  &  http://edcftp.cr.usgs.gov/pub/data/gtopo30/global \\
 GTOPO30\_LIS     & Similar dataset as to the above one but processed by LIS-team.\\
 GTOPO30\_GFS     & Similar dataset as to the above but on GFS grid.\\
 SRTM\_Native     & The SRTM elevation map native source. See: \\
                  &  http://dds.cr.usgs.gov/srtm/version2\_1/SRTM30 \\
 SRTM\_LIS        & Similar dataset as to the above one but processed by LIS-team.\\
 CONSTANT         & User can set a constant elevation, slope or aspect class. \\
 \end{tabular}
 

 \begin{Verbatim}[frame=single]
Elevation data source:  "SRTM_Native"
Slope data source:      "SRTM_Native"
Aspect data source:     "SRTM_Native"
Curvature data source:  "SRTM_Native"
 \end{Verbatim}

 
 \var{Elevation number of bands:} specifies the number of
 elevation bands or bins to turn on elevation tiling capability.

 \var{Slope number of bands:} specifies the number of
 slope bands or bins to turn on slope tiling capability.

 \var{Aspect number of bands:} specifies the number of
 aspect bands or bins to turn on aspect tiling capability.

 \var{Curvature number of bands:} specifies the number of
 curvature bands or bins to turn on curvature tiling capability.

 

 \begin{Verbatim}[frame=single]
Elevation number of bands:     1
Slope number of bands:         1
Aspect number of bands:        1
Curvature number of bands:     1
 \end{Verbatim}

 
 Topography maps

 \var{Elevation map:} specifies the elevation of the LIS grid.
  If the elevation map type selected is SRTM\_Native, then the
   elevation file entry is actually just the directory path, which
   contains the tiled SRTM elevation files.

 \var{Slope map:} specifies the slope of the LIS grid.
  If the slope map type selected is SRTM\_Native, then the
   file entry is actually just the directory path, which
   contains the tiled SRTM elevation files.

 \var{Aspect map:} specifies the aspect of the LIS grid.
  If the aspect map type selected is SRTM\_Native, then the
   file entry is actually just the directory path, which
   contains the tiled SRTM elevation files.

 \var{Curvature map:} specifies the curvature of the LIS grid.
 

 \begin{Verbatim}[frame=single]
Elevation map:     ../input/25KM/elev_GTOPO30.1gd4r
Slope map:         ../input/25KM/slope_GTOPO30.1gd4r
Aspect map:        ../input/25KM/aspect_GTOPO30.1gd4r
Curvature map:     ../input/25KM/curv_GTOPO30.1gd4r
 \end{Verbatim}

 
 \var{Elevation fill option:} specifies the elevation
 data fill option.  Options include:

 \begin{tabular}{ll}
 Value    & Description                                \\
 none     & Do not apply missing value fill routines   \\
 neighbor & Use nearest neighbor to fill missing value \\
 \end{tabular}

 \var{Elevation fill value:} indicates which elevation
 value to be used if an arbitrary value fill is needed. 
 (For example, when the landmask indicates a land point but no existing 
 elevation value, a value of 100(m) could be assigned if 
 no nearest neighbor values exists to fill).

 \var{Elevation fill radius:} specifies the radius with which
 to search for nearby value(s) to help fill in the missing value.
 

 \begin{Verbatim}[frame=single]
Elevation fill option:          neighbor
Elevation fill radius:           2.
Elevation fill value:           100.
 \end{Verbatim}

 
 \var{Slope fill option:} specifies the slope
 data fill option.  Options include:

 \begin{tabular}{ll}
 Value    & Description                                \\
 none     & Do not apply missing value fill routines   \\
 neighbor & Use nearest neighbor to fill missing value \\
 \end{tabular}

 \var{Slope fill value:} indicates which slope
 value to be used if an arbitrary value fill is needed. 
 (For example, when the landmask indicates a land point but no existing 
 slope value, an value of 0.1 could be assigned if 
 no nearest neighbor values exists to fill).

 \var{Slope fill radius:} specifies the radius with which
 to search for nearby value(s) to help fill in the missing value.
 

 \begin{Verbatim}[frame=single]
Slope fill option:         neighbor
Slope fill radius:           2.
Slope fill value:           0.1
 \end{Verbatim}

 
 \var{Aspect fill option:} specifies the aspect
 data fill option.  Options include:

 \begin{tabular}{ll}
 Value    & Description                                \\
 none     & Do not apply missing value fill routines   \\
 neighbor & Use nearest neighbor to fill missing value \\
 \end{tabular}

 \var{Aspect fill value:} indicates which aspect
 value to be used if an arbitrary value fill is needed. 
 (For example, when the landmask indicates a land point but no existing 
 aspect value, an value of 2.0 could be assigned if 
 no nearest neighbor values exists to fill).

 \var{Aspect fill radius:} specifies the radius with which
 to search for nearby value(s) to help fill in the missing value.
 

 \begin{Verbatim}[frame=single]
Aspect fill option:        neighbor
Aspect fill radius:           2.
Aspect fill value:           2.0
 \end{Verbatim}

 
 \var{Topography map projection:} specifies the projection of the
 topography map data.

 \var{Topography spatial transform:} indicates which spatial transform
 (i.e., upscale or downscale) type is to be applied to the topographic
 map.  Options include:

 \begin{tabular}{ll}
 Value     & Description                                         \\
 none      & Data is on same grid as LIS output domain            \\
 average   & Upscale by averaging values for each gridcell       \\
 neighbor  & Reinterpolate by selecting nearest gridcell neighbor \\
 bilinear  & Reinterpolate by using bilinear interpolation        \\
 budget-bilinear & Reinterpolate by using conservative, budget-bilinear \\
 tile      &  Read in tiled data or upscale by estimating gridcell
              fractions                                            \\
 \end{tabular}
 

 \begin{Verbatim}[frame=single]
Topography spatial transform:     tile
 \end{Verbatim}

 
 This section should also specify the domain specifications of the
 topography data.
 If the map projection of parameter data is specified to be lat/lon,
 the following configuration should be used for specifying topography
 data, especially if the data source option has a ``\_LIS'' in the name.
 See Appendix~\ref{sec:d_latlon_example} for more details about
 setting these values.
 

 \begin{Verbatim}[frame=single]
Topography map projection:         latlon
Topography lower left lat:       -59.87500
Topography lower left lon:      -179.87500
Topography upper right lat:       89.87500
Topography upper right lon:      179.87500
Topography resolution (dx):        0.2500
Topography resolution (dy):        0.2500
 \end{Verbatim}


 
 \subsection{LSM-specific Parameters} \label{ssec:lsmspecparams}
 


 
 Albedo maps

 \var{Albedo data source:} specifies the albedo climatology map
 dataset source to be read in. Current source options include:

 \begin{tabular}{ll}
 Value         & Description    \\
 NCEP\_Native  &  Native monthly NCEP albedo files. \\
 NCEP\_LIS     &  Similar to the above option but LIS-team processed. \\
 CONSTANT      &  User can select a constant value. \\
 \end{tabular}

 \var{Albedo map:} specifies the path of the climatology based
 albedo files.  The climatology albedo data files have the following
 naming convention: $<$directory$>$/$<$file header$>$.$<$tag$>$.1gd4r
 The tag should be either sum, win, spr, or aut depending on the season,
 or the tag should represent the month (such as jan, feb, mar, etc.).
 The file header can be anything (such as alb1KM).
  The albedo field is used by Noah LSM versions.

 \var{Albedo map projection:} specifies the projection of the
 albedo map data.

 \var{Albedo climatology interval:} specifies the frequency of the
 albedo climatology in months.

 \begin{tabular}{ll}
 Value     & Description                        \\
 monthly   & Monthly interval for albedo files  \\
 quarterly & Seasonal interval for albedo files \\
 \end{tabular}

 \var{Albedo spatial transform:} indicates which spatial transform
 (i.e., upscale or downscale) type is to be applied to the albedo
 maps.  Options include:

 \begin{tabular}{ll}
 Value           & Description                                          \\
 none            & Data is on same grid as LIS output domain            \\
 average         & Upscale by averaging values for each gridcell        \\
 neighbor        & Reinterpolate by selecting nearest gridcell neighbor \\
 bilinear        & Reinterpolate by using bilinear interpolation        \\
 budget-bilinear & Reinterpolate by using conservative, budget-bilinear \\
 \end{tabular}
 

 \begin{Verbatim}[frame=single]
Albedo data source:           NCEP_LIS
Albedo map:                ../input/25KM/albedo_NCEP 
Albedo climatology interval:  monthly  
Albedo spatial transform:     none
 \end{Verbatim}

 
 If selecting the Catchment LSM (F2.5 version), the model requires the
 near infrared (NIR) and visible (VIS) albedo factor files, as shown
 below for example. 
 This particular albedo parameter set is currently only available for the
 Catchment LSM Fortuna 2.5 (CLSMF2.5).

 \var{Albedo NIR factor file:} specifies the NIR albedo factor file. 

 \var{Albedo VIS factor file:} specifies the VIS albedo factor file. 

 These albedo parameter subroutines can be found in the albedo directory.
  

 \begin{Verbatim}[frame=single]
Albedo NIR factor file: ./GLDAS_1.0-deg/modis_scale_factor.albnf.clim
Albedo VIS factor file: ./GLDAS_1.0-deg/modis_scale_factor.albvf.clim
 \end{Verbatim}

 
 \var{Albedo fill option:} specifies the albedo
 data fill option.  Options include:

 \begin{tabular}{ll}
 Value   & Description                              \\
 none    & Do not apply missing value fill routines \\
 average & Use average to fill missing value        \\
 \end{tabular}

 \var{Albedo fill value:} indicates which albedo
 value to be used if an arbitrary value fill is needed. 
(For example, when the landmask indicates a land point but no existing 
 albedo value, a value of 0.12 could be assigned if 
 no nearest neighbor values exists to fill).

 \var{Albedo fill radius:} specifies the radius with which
 to search for nearby value(s) to help fill in the missing value.
 

 \begin{Verbatim}[frame=single]
Albedo fill option:            average
Albedo fill radius:               2.
Albedo fill value:               0.12
 \end{Verbatim}

 
 If the map projection of parameter data is specified to be lat/lon, 
 the following configuration should be used for specifying albedo data
 where the albedo data source option has a ``\_LIS'' in the name.
 See Appendix~\ref{sec:d_latlon_example} for more details about
 setting these values.
  

 \begin{Verbatim}[frame=single]
Albedo map projection:       latlon
Albedo lower left lat:      -59.87500
Albedo lower left lon:     -179.87500
Albedo upper right lat:      89.87500
Albedo upper right lon:     179.87500
Albedo resolution (dx):       0.2500
Albedo resolution (dy):       0.2500
 \end{Verbatim}

 
 \var{Max snow albedo data source:} specifies the maximum snow albedo
 dataset source to be read in. Current source options include:

 \begin{tabular}{ll}
 Value         & Description    \\
 NCEP\_Native  &  Native NCEP maximum snow albedo source. \\
 NCEP\_LIS     &  Similar to the above option but LIS-team processed. \\
 NCEP\_GFS     &  Similar to the above option but on GFS grid. \\
 SACHTET.3.5.6 &  Max snow albedo specific to the SAC-HTET model. \\
 CONSTANT      &  User can select a constant value. \\
 \end{tabular}

 \var{Max snow albedo map:} specifies the map file containing
 data with the static upper bound of the snow albedo.
  The albedo field is used by all Noah LSM and RDHM-SAC LSM versions.

 \var{Max snow albedo map projection:} specifies the projection of the
 max snow albedo map data.
 \var{Max snow albedo spatial transform:} indicates which spatial
 transform (i.e., upscale or downscale) type is to be applied
 to the maximum snow albedo map.  Options include:

 \begin{tabular}{ll}
 Value    & Description                                             \\
 none      & Data is on same grid as LIS output domain              \\
 average   & Upscale by averaging values for each gridcell          \\
 neighbor  & Reinterpolate by selecting nearest gridcell neighbor   \\
 bilinear  & Reinterpolate by using bilinear interpolation          \\
 budget-bilinear & Reinterpolate by using conservative, budget-bilinear \\
 \end{tabular}
 

 \begin{Verbatim}[frame=single]
Max snow albedo data source:       NCEP_LIS
Max snow albedo map:   ../input/25KM/mxsnoalb_MODIS.1gd4r
Max snow albedo spatial transform:  none
 \end{Verbatim}

 
 \var{Max snow albedo fill option:} specifies the max snow albedo
 data fill option.  Options include:

 \begin{tabular}{ll}
 Value   & Description                               \\
 none    &  Do not apply missing value fill routines \\
 average &  Use average to fill missing value        \\
 \end{tabular}

 \var{Max snow albedo fill value:} indicates which max snow albedo
 value to be used if an arbitrary value fill is needed. 
 (For example, when the landmask indicates a land point but no existing 
 snow albedo value, an value of 0.42 could be assigned if 
 no nearest neighbor values exists to fill).

 \var{Max snow albedo fill radius:} specifies the radius with which
 to search for nearby value(s) to help fill in the missing value.
 

 \begin{Verbatim}[frame=single]
Max snow albedo fill option:          average
Max snow albedo fill radius:             3.
Max snow albedo fill value:             0.42
 \end{Verbatim}

 
 If the map projection of parameter data is specified to be lat/lon,
 the following configuration should be used for specifying max snow 
 albedo data, where the max snow albedo
 albedo data source option has a ``\_LIS'' in the name.
 See Appendix~\ref{sec:d_latlon_example} for more details about
 setting these values.
 

 \begin{Verbatim}[frame=single]
Max snow albedo map projection:      latlon
Max snow albedo lower left lat:    -59.87500
Max snow albedo lower left lon:   -179.87500
Max snow albedo upper right lat:    89.87500
Max snow albedo upper right lon:   179.87500
Max snow albedo resolution (dx):     0.2500
Max snow albedo resolution (dy):     0.2500
 \end{Verbatim}

 
 Greenness fraction maps

 Greenness vegetation fraction is considered the horizontal greenness
 fraction represented for a model gridcell. This parameter is used in the
 LSMs: all Noah LSMs, RDHM-SAC, Catchment F2.5. 

 \var{Greenness data source:} specifies the greenness fraction
 climatology dataset source to be read in. Current source options include:

 \begin{tabular}{ll}
 Value         & Description    \\
 NCEP\_Native  &  Native NCEP monthly greenness climatology source. \\
 NCEP\_LIS     &  Similar to the above option but LIS-team processed. \\
 CLSMF2.5      &  Similar to the above option but for CLSM F2.5 model. \\
 SACHTET.3.5.6 &  Similar to the above option but for SAC-HTET model. \\
 CONSTANT      &  User can select a constant value. \\
 \end{tabular}

 \var{Greenness map projection:} specifies the projection of the
 greenness map data.

 \var{Greenness fraction map:} specifies the source of the
 climatology based gfrac files.  The climatology greenness data
 files have the following naming convention:
 $<$directory$>$/$<$file header$>$.$<$tag$>$.1gd4r.
 The tag should represent the month (such as jan, feb, mar, etc.).
 The file header can be anything (such as green1KM).

 \var{Greenness climatology interval:} specifies the frequency of
 the greenness climatology in months.
 Only current option is: ``monthly''.

 \var{Calculate min-max greenness fraction:} specifies a
 logical flag option to offer the user the ability to calculate
 minimum and maximum greenness fraction values from a given
 climatology (e.g., monthly).
 Acceptable values are:

 \begin{tabular}{ll}
 Value     & Description                 \\
 .false.   & Read in min and max greenness fraction value maps     \\
 .true.    & Calculate greenness fraction from greenness climatology maps \\
 \end{tabular}

 \var{Greenness maximum map:} specifies the file of the
 climatological maximum greenness data from the monthly
 greenness files.

 \var{Greenness minimum map:} specifies the file of the
 climatological minimum greenness data from the monthly
 greenness files.

 \var{Greenness spatial transform:} indicates which spatial transform
 (i.e., upscale or downscale) type is to be applied to the greenness
 maps.  Options include:

 \begin{tabular}{ll}
 Value     & Description                                           \\
 none       & Data is on same grid as LIS output domain            \\
 average    & Upscale by averaging values for each gridcell        \\
 neighbor   & Reinterpolate by selecting nearest gridcell neighbor \\
 bilinear   & Reinterpolate by using bilinear interpolation        \\
 budget-bilinear & Reinterpolate by using conservative, budget-bilinear \\
 \end{tabular}
 

 \begin{Verbatim}[frame=single]
Greenness data source:         NCEP_LIS
Greenness fraction map:     ../input/25KM/gvf_NCEP
Greenness climatology interval:   monthly
Calculate min-max greenness fraction:  .true.
Greenness maximum map:      ../input/25KM/gvf_NCEP.MAX.1gd4r
Greenness minimum map:      ../input/25KM/gvf_NCEP.MIN.1gd4r
Greenness spatial transform:     none
 \end{Verbatim}

 
 \var{Greenness fill option:} specifies the greenness fraction
 data fill option.  Options include:

 \begin{tabular}{ll}
 Value    & Description                               \\
 none     &  Do not apply missing value fill routines \\
 average  &  Use average to fill missing value        \\
 \end{tabular}

 \var{Greenness fill radius:} specifies the radius with which
 to search for nearby value(s) to help fill in the missing value.

 \var{Greenness fill value:} indicates which greenness fraction 
 value to be used if an arbitrary value fill is needed. 
 (For example, when the landmask indicates a land point but no existing 
 greenness value, a value of 0.2 could be assigned if 
 exists to fill).

 \var{Greenness maximum fill value:} indicates which maximum greenness
 fraction value to be used if an arbitrary value fill is needed. 

 \var{Greenness minimum fill value:} indicates which minimum greenness
 fraction value to be used if an arbitrary value fill is needed. 
 

 \begin{Verbatim}[frame=single]
Greenness fill option:        average
Greenness fill radius:           3
Greenness fill value:           0.20
Greenness maximum fill value:   0.80
Greenness minimum fill value:   0.05
 \end{Verbatim}

 
 If the map projection of parameter data is specified to be lat/lon, 
 the following configuration should be used for specifying greenness
 data source, if the option has a ``\_LIS'' in the name.
 See Appendix~\ref{sec:d_latlon_example} for more details about
 setting these values.
 

 \begin{Verbatim}[frame=single]
Greenness map projection:        latlon
Greenness lower left lat:      -59.87500
Greenness lower left lon:     -179.87500
Greenness upper right lat:      89.87500
Greenness upper right lon:     179.87500
Greenness resolution (dx):       0.2500
Greenness resolution (dy):       0.2500
 \end{Verbatim}

 
 LAI/SAI maps
 Leaf area index and stem area index maps are used to describe the 
 vertical representation of leafy vegetation and the woody-branch
 areas within a given gridecell (respectively).  
 LAI/SAI are used in the Community Land Model (CLM), Mosaic LSM, and
 Catchment LSM, version F2.5.


 \var{LAI/SAI map projection:} specifies the projection of the
 LAI/SAI map data.

 \var{LAI data source:} specifies the leaf area index (LAI)
 climatology dataset source to be read in. Current source options include:

 \begin{tabular}{ll}
 Value      & Description    \\
 AVHRR       &  LIS-team produced monthly LAI climatology source. \\
 CLSMF2.5    &  Similar to the above option but for CLSM F2.5 model. \\
 CONSTANT    &  User can select a constant value. \\
 \end{tabular}

 \var{SAI data source:} specifies the stem area index (SAI)
 climatology dataset source to be read in. Current source options include:

 \begin{tabular}{ll}
 Value      & Description    \\
 AVHRR       &  LIS-team produced monthly SAI climatology source. \\
 CONSTANT    &   User can select a constant value. \\
 \end{tabular}

 \var{LAI map:} specifies the source of the climatology based
 LAI files. The climatology data files have the following
 naming convention: $<$directory$>$/$<$file header$>$.$<$tag$>$.1gd4r.
 The tag should be represent the month (such as jan, feb, mar, etc.).
 The file header can be anything (such as avhrr\_lai\_1KM).

 \var{SAI map:} specifies the source of the climatology based
 SAI files. The climatology data files have the following
 naming convention: $<$directory$>$/$<$file header$>$.$<$tag$>$.1gd4r.
 The tag should be represent the month (such as jan, feb, mar, etc.).
 The file header can be anything (such as avhrr\_sai\_1KM).

 \var{LAI/SAI climatology interval:} specifies the frequency of the
 LAI or SAI climatology in months. Current option is:  ``monthly''.

 \var{Calculate min-max LAI:} specifies a
 logical flag option to offer the user the ability to calculate
 minimum and maximum LAI values from a given climatology (e.g., monthly).
 Acceptable values are:

 \begin{tabular}{ll}
 Value     & Description                                \\
 .false.   & Read in min and max LAI value maps         \\
 .true.    & Calculate LAI from LAI climatology maps    \\
 \end{tabular}

 \var{LAI maximum map:} specifies the file of the
 climatological maximum LAI data from the monthly
 LAI files.

 \var{LAI minimum map:} specifies the file of the
 climatological minimum LAI data from the monthly
 LAI files.

 \var{LAI/SAI spatial transform:} indicates which spatial transform
 (i.e., upscale or downscale) type is to be applied to the LAI and SAI
 maps.  Only ``none'' option works for the ``AVHRR'' or ``CLSMF2.5'' 
 LAI data source entries.  Other spatial options for the include:

 \begin{tabular}{ll}
 Value   & Description                                   \\
 none    & Data is on same grid as LIS output domain     \\
 average         & Upscale by averaging values for each gridcell        \\
 neighbor        & Reinterpolate by selecting nearest gridcell neighbor \\
 bilinear        & Reinterpolate by using bilinear interpolation        \\
 budget-bilinear & Reinterpolate by using conservative, budget-bilinear \\
 \end{tabular}
 

 \begin{Verbatim}[frame=single]
LAI data source:            CLSMF2.5
LAI map:          ../input/25KM/avhrr_lai_nldas               
SAI map:          ../input/25KM/avhrr_sai_nldas           
Calculate min-max LAI:      .false.
LAI maximum map:  ../input/CLSMF2.5/clsmf2.5_maxlai.1gd4r
LAI minimum map:  ../input/CLSMF2.5/clsmf2.5_minlai.1gd4r
LAI/SAI climatology interval:  monthly 
LAI/SAI spatial transform:     none
 \end{Verbatim}

 
 \var{LAI/SAI fill option:} specifies the LAI/SAI
 data fill option.  Options include:

 \begin{tabular}{ll}
 Value   & Description                              \\
 none    & Do not apply missing value fill routines \\
 average & Use average to fill missing value        \\
 \end{tabular}

 \var{LAI/SAI fill radius:} specifies the radius with which
 to search for nearby value(s) to help fill in the missing value.

 \var{LAI fill value:} indicates which LAI 
 value to be used if an arbitrary value fill is needed. 
 (For example, when the landmask indicates a land point but no existing 
 LAI value, a value of 1 could be assigned if 
 exists to fill).
  
 \var{LAI maximum fill value:} indicates which maximum LAI 
 value to be used if an arbitrary value fill is needed. 
 
 \var{LAI minimum fill value:} indicates which minimum LAI 
 value to be used if an arbitrary value fill is needed. 

 \var{SAI fill value:} indicates which SAI
 value to be used if an arbitrary value fill is needed. 
 \nextpar
 

 \begin{Verbatim}[frame=single]
LAI/SAI fill option:     average
LAI/SAI fill radius:        3 
LAI fill value:             1
SAI fill value:            0.5
LAI maximum fill value:     4
LAI minimum fill value:     1
 \end{Verbatim}

 
 If the map projection of parameter data is specified to be lat/lon, 
 the following configuration should be used for specifying LAI/SAI data,
 if  the data source option has a ``\_LIS'' in the name.
 See Appendix~\ref{sec:d_latlon_example} for more details about
 setting these values.
 

 \begin{Verbatim}[frame=single]
LAI/SAI map projection:        latlon
LAI/SAI lower left lat:      -59.87500
LAI/SAI lower left lon:     -179.87500
LAI/SAI upper right lat:      89.87500
LAI/SAI upper right lon:     179.87500
LAI/SAI resolution (dx):       0.2500
LAI/SAI resolution (dy):       0.2500
 \end{Verbatim}


 
 \var{Slope type data source:} specifies the slope type index 
 dataset source to be read in. Current source options include:

 \begin{tabular}{ll}
 Value       & Description    \\
 NCEP\_Native  &  Native NCEP slope type derived map source. \\
 NCEP\_LIS     &  Similar to the above option but LIS-team processed. \\
 NCEP\_GFS     &  Similar to the above option but on a GFS grid type. \\
 CONSTANT      &  User can select a constant value.  \\
 \end{tabular}

 \var{Slope type map:} specifies the slope type index as used in
 all Noah LSM versions.

 \var{Slope type map projection:} specifies the projection of the
 slope type map data.

 \var{Slope type spatial transform:} indicates which spatial transform
 (i.e., upscale or downscale) type is to be applied to the soils
 maps.  Options include:

 \begin{tabular}{ll}
 Value    & Description                                          \\
 none     & Data is on same grid as LIS output domain            \\
 mode     & Upscale by selecting dominant type for each gridcell \\
 neighbor & Use nearest neightbor to select nearest gridcell neighbor \\
 \end{tabular}
 

 \begin{Verbatim}[frame=single]
Slope type data source:        NCEP_LIS
Slope type map:         ../input/25KM/slopetype_NCEP.1gd4r
Slope type spatial transform:   none
 \end{Verbatim}

 
 \var{Slope type fill option:} specifies the slope type 
 data fill option.  Options include:

 \begin{tabular}{ll}
 Value    & Description                                \\
 none     & Do not apply missing value fill routines   \\
 neighbor & Use nearest neighbor to fill missing value \\
 \end{tabular}

 \var{Slope type fill value:} indicates which slope type 
 value to be used if an arbitrary value fill is needed. 
 (For example, when the landmask indicates a land point but no existing 
 slope type value, an index value of 1 could be assigned if 
 no nearest neighbor values exists to fill).

 \var{Slope type fill radius:} specifies the radius with which
 to search for nearby value(s) to help fill in the missing value.
 

 \begin{Verbatim}[frame=single]
Slope type fill option:        neighbor
Slope type fill radius:         2.
Slope type fill value:          1.
 \end{Verbatim}

 
 If the map projection of parameter data is specified to be lat/lon, 
 the following configuration should be used for specifying slope type
 data, if the data source option has a ``\_LIS'' in the name.
 See Appendix~\ref{sec:d_latlon_example} for more details about
 setting these values.
 

 \begin{Verbatim}[frame=single]
Slope type map projection:       latlon
Slope type lower left lat:      -59.87500
Slope type lower left lon:     -179.87500
Slope type upper right lat:      89.87500
Slope type upper right lon:     179.87500
Slope type resolution (dx):       0.2500
Slope type resolution (dy):       0.2500
 \end{Verbatim}

 

 \var{Bottom temperature data source:} specifies the bottom temperature
 dataset source to be read in. Current source options include:

 \begin{tabular}{ll}
 Value         & Description    \\
 ISLSCP1       &  Native (NCEP) ISLSCP1 temperature derived map. \\
 NCEP\_LIS     &  Similar to the above option but LIS-team processed. \\
 NCEP\_GFS     &  Similar to the above option but on a GFS grid type. \\
 CONSTANT      &  User can select a constant value. \\
 \end{tabular}

 \var{Bottom temperature map:} specifies the bottom boundary
 temperature data.
 This parameter is currently required by the Noah LSM versions
  and the recently added RDHM-SAC/Snow-17 models.

 \var{Bottom temperature map projection:} specifies the projection of the
 bottom temperature map data.

 \var{Bottom temperature spatial transform:} indicates which spatial
 transform (i.e., upscale or downscale) type is to be applied to 
 the bottom temperature map.  Options include:

 \begin{tabular}{ll}
 Value           & Description                                   \\
 none            & Data is on same grid as LIS output domain     \\
 average         & Upscale by averaging values for each gridcell \\
 neighbor        & Nearest neighbor scheme                       \\
 bilinear        & bilinear scheme                               \\
 budget-bilinear & conservative scheme                           \\
 \end{tabular}
 

 \begin{Verbatim}[frame=single]
Bottom temperature data source:       NCEP_LIS
Bottom temperature map:  ../input/25KM/tbot_GDAS_6YR_CLIM.1gd4r
Bottom temperature spatial transform:   none
 \end{Verbatim}

 
 \var{Bottom temperature fill option:} specifies the bottom boundary
 temperature data fill option.  Options include:

 \begin{tabular}{ll}
 Value    & Description                                \\
 none     & Do not apply missing value fill routines   \\
 average  & Averaging values for each missing value    \\
 neighbor & Use nearest neighbor to fill missing value \\
 \end{tabular}

 \var{Bottom temperature fill value:} indicates which bottom soil 
 temperature value to be used if an arbitrary value fill is needed. 
 (For example, when the landmask indicates a land point but no existing 
 bottom temperature field, a value of 287 K could be assigned if 
 no nearest neighbor values exists to fill).

 \var{Bottom temperature fill radius:} specifies the radius with which
 to search for nearby value(s) to help fill in the missing value.
 

 \begin{Verbatim}[frame=single]
Bottom temperature fill option:   neighbor    
Bottom temperature fill radius:     3.      
Bottom temperature fill value:     287.0      
 \end{Verbatim}

 
 \var{Bottom temperature topographic downscaling:} specifies the
 option with which to adjust bottom temperature field due to
 topographic impacts.

 \begin{tabular}{ll}
 Value       & Description                  \\
 none        & No topographic/elevation adjustment made to 
               parameter data \\
 lapse-rate  & Adjust (or downscale) bottom temperature using
               lapse-rate correction. \\
 \end{tabular}
 

 \begin{Verbatim}[frame=single]
Bottom temperature topographic downscaling:    none
 \end{Verbatim}

 
 If the map projection of parameter data is specified to be lat/lon, 
 the following configuration should be used for specifying bottom
 temperature parameter data, if the data source option has a 
 ``\_LIS'' in the name.
 See Appendix~\ref{sec:d_latlon_example} for more details about
 setting these values.
 

 \begin{Verbatim}[frame=single]
Bottom temperature map projection:      latlon 
Bottom temperature lower left lat:     -59.87500
Bottom temperature lower left lon:    -179.87500
Bottom temperature upper right lat:     89.87500
Bottom temperature upper right lon:    179.87500
Bottom temperature resolution (dx):      0.2500
Bottom temperature resolution (dy):      0.2500
 \end{Verbatim}

 
 \var{Noah-MP PBL Height Value:} specifies the
 option which to set the planetary boundary layer
 height (PBLH) value for the Noah-MP (3.6) model.

 

 \begin{Verbatim}[frame=single]
Noah-MP PBL Height Value:     900.    # in meters
 \end{Verbatim}


 
 Potential Evapotranspiration (PET) maps

 \var{PET directory:} specifies the source of the monthly climatology based
 PET files. The climatology data files have the following
 naming convention: $<$directory$>$/$<$file header$>$.$<$tag$>$.1gd4r.
 The tag should be represent the month (such as JAN, FEB, MAR, etc.).
 The file header can be anything (such as avhrr\_pet\_1KM).
 Currently, this parameter is used only with the RDHM-SAC model.

 \var{PET map projection:} specifies the projection of the
 PET map data.

 \var{PET adjustment factor directory:} specifies the source of the m
 monthly climatology-based PET adjustment factor files.
 The climatology data files have the following
 naming convention: $<$directory$>$/$<$file header$>$.$<$tag$>$.1gd4r.
 The tag should be represent the month (such as JAN, FEB, MAR, etc.).
 The file header can be anything (such as avhrr\_petadj\_1KM).

 \var{PET climatology interval:} specifies the frequency of the
  PET climatology in months. Current option is:  ``monthly''.

 \var{PET spatial transform:} indicates which spatial transform
 (i.e., upscale or downscale) type is to be applied to the PET 
 maps.  Options include:

 \begin{tabular}{ll}
 Value   & Description                                   \\
 none    & Data is on same grid as LIS output domain (only option for now) \\
 \end{tabular}
 

 \begin{Verbatim}[frame=single]
PET directory:             ../input/25KM/sachtet_pet
PET adjustment factor directory:  ../input/25KM/sachtet_petadj
PET climatology interval:    monthly
PET spatial transform:        none
 \end{Verbatim}

 
 \var{PET fill option:} specifies the PET climatology
 data fill option.  Options include:

 \begin{tabular}{ll}
 Value   & Description                              \\
 none    & Do not apply missing value fill routines \\
 average & Use average to fill missing value        \\
 \end{tabular}

 \var{PET fill radius:} specifies the radius with which
 to search for nearby value(s) to help fill in the missing value.

 \var{PET fill value:} indicates which  PET
 value to be used if an arbitrary value fill is needed. 
 (For example, when the landmask indicates a land point but no existing 
 PET value, a value of 1 could be assigned if exists to fill.
 \nextpar
 

 \begin{Verbatim}[frame=single]
PET fill option:         average
PET fill radius:            3
PET fill value:            10.
 \end{Verbatim}

 
 If the map projection of parameter data is specified to be lat/lon, 
 the following configuration should be used for specifying PET data.
 See Appendix~\ref{sec:d_latlon_example} for more details about
 setting these values.
 

 \begin{Verbatim}[frame=single]
PET map projection:        latlon
PET lower left lat:      -59.87500
PET lower left lon:     -179.87500
PET upper right lat:      89.87500
PET upper right lon:     179.87500
PET resolution (dx):       0.2500
PET resolution (dy):       0.2500
 \end{Verbatim}

 
 \var{CLSMF25 map projection:} specifies the projection of the
 CLSMF25 map data.
 

 
 \var{CLSMF25 tile coord file:} specifies the location of a CLSM F2.5
 coordinate file.  This file contains catchment tile coordinate
 information that can be used in
 Catchment LSM (CLSM) Fortuna 2.5 version model run.
 

 \begin{Verbatim}[frame=single]
CLSMF25 tile coord file:  ./cat_parms/PE_2880x1440_DE_464x224.file
 \end{Verbatim}

 
 \var{CLSMF25 soil param file:} specifies the location of a
 CLSM F2.5 soils file.  This file contains catchment soil parameter
 information that can be used in Catchment LSM (CLSM) Fortuna 2.5
 version model run.
 

 \begin{Verbatim}[frame=single]
CLSMF25 soil param file:  ./cat_parms/soil_param.dat
 \end{Verbatim}

 
 \var{CLSMF25 topo files:} specifies the locations of a CLSM F2.5
 topo parameter files.  These files contain catchment topographic
 parameter information that can be used in a Catchment LSM (CLSM)
 Fortuna 2.5 version model run.

 \var{CLSMF25 topo ar file:} specifies the table file containing 
   topographic shape parameters for the CLSM F2.5 model.
 \var{CLSMF25 topo bf file:} specifies the table file containing
   topographic baseflow paramters for the CLSM F2.5 model.
 \var{CLSMF25 topo ts file:} specifies the table file containing
   water transfer timescale parameters for the CLSM F2.5 model.

 

 \begin{Verbatim}[frame=single]
CLSMF25 topo ar file:  ../cat_parms/ar.new
CLSMF25 topo bf file:  ../cat_parms/bf.dat
CLSMF25 topo ts file:  ../cat_parms/ts.dat
 \end{Verbatim}

 
 \var{CLSMF25 surf layer ts file:} specifies the location of a
 CLSM F2.5 tau parameter file.  This file contain catchment surface
 layer timescale (ts), tau, parameter information that can be used
 in Catchment LSM (CLSM) Fortuna 2.5 version model runs.
 

 \begin{Verbatim}[frame=single]
CLSMF25 surf layer ts file:  ../cat_parms/tau_param.dat
 \end{Verbatim}

 
 \var{CLSMF25 top soil layer depth:} specifies the top soil layer
 depth.  This parameter value specifies the depth of the top soil
 layer depth (unit: meters) and is needed in processing other
 parameters for a Catchment LSM (CLSM) Fortuna 2.5 version model run.
 

 \begin{Verbatim}[frame=single]
CLSMF25 top soil layer depth:   0.02 
 \end{Verbatim}

 
 \var{CLSMF25 spatial transform:} indicates which spatial transform
 (i.e., upscale or downscale) type is to be applied to CLSM F2.5
 parameters.
 Options include (only 'none' works at this time):

 \begin{tabular}{ll}
 Value   & Description                  \\
 none    &  Data is on same grid as LIS output domain \\
 \end{tabular}
 
 \begin{Verbatim}[frame=single]
CLSMF25 spatial transform:     none
 \end{Verbatim}

 
 This section also outlines the domain specifications of the
 Catchment LSM Fortuna 2.5 data.
 If the map projection of parameter data is specified to be lat/lon,
 the following configuration should be used for specifying CLSM data.
 See Appendix~\ref{sec:d_latlon_example} for more details about
 setting these values.
 
 \begin{Verbatim}[frame=single]
CLSMF25 map projection:       latlon
CLSMF25 lower left lat:       25.0625
CLSMF25 lower left lon:     -124.9375
CLSMF25 upper right lat:      52.9375
CLSMF25 upper right lon:     -67.0625
CLSMF25 resolution (dx):       0.125
CLSMF25 resolution (dy):       0.125
 \end{Verbatim}

 
 \var{RDHM356 constants table:} specifies the location of
  the constants table required by the Research Distributed 
  Hydrologic Model (RDHM) version 3.5.6 models, SAC-HTET and SNOW-17.
  This table file contains constant values for any listed SAC-HTET
  or SNOW-17 parameter types.  If a constant value is >= 0., then
  the constant value is assigned for all gridcells for a parameter entry.
  If the value is negative, a 2-D gridded a priori map is read in.
  Also, the negative constant value can be used as a scaling factor
  of the 2-D grid by taking its absolute value and multiplying the
  entire field by it, if the value is other than -1.

 \var{RDHM356 universal undefined value:} specifies an universal
  undefined value that can be used by either the SAC-HTET or SNOW-17
  models for run-time purposes.
 
 \begin{Verbatim}[frame=single]
RDHM356 constants table:   ./rdhm_singlevalueinputs.txt
RDHM356 universal undefined value:  -1.
 \end{Verbatim}

 
 \var{Create or readin soil parameters:} specifies how the 
 soil parameter files are either generated or brought in to
 the SAC-HTET model.
 Options include:

 \begin{tabular}{ll}
 Value   & Description                  \\
 none    &  do not readin or create soil parameters \\
 readin  &  read in existing SAC soil parameter files \\
 create  &  generate SAC soil parameter fields in LDT \\
         &  (currently only available at native STATSGOv1 grid\\
         &  at the lat-lon grid and 0.00833 deg resolution.  \\
 \end{tabular}
 
 \begin{Verbatim}[frame=single]
Create or readin soil parameters:     "readin"   
 \end{Verbatim}

 
 \var{SACHTET soil parameter method:} specifies the method
  that can generate the SAC soil parameters.
 Options include (for now):

 \begin{tabular}{ll}
 Value     & Description                  \\
 none      &  do not readin or create soil parameters \\
 Koren\_v1 & Based on Victor Koren (NOAA/OHD) original code\\
           &  developed to generate SAC soil parameters.\\
 \end{tabular}
 
 \begin{Verbatim}[frame=single]
SACHTET soil parameter method:        "Koren_v1"   # none | Koren_v1
 \end{Verbatim}

 
 \var{SACHTET Cosby soil parameter table:} specifies the path 
  of the Cosby soil parameter table needed for the SAC-HTET 
  soil parameters, especially for the generation of the parameters.
 
 \begin{Verbatim}[frame=single]
SACHTET Cosby soil parameter table:  ./rdhm_parms/cosby_eq_newzperc.txt
 \end{Verbatim}

 
 \var{SACHTET parameter files:} specifies the locations of SACHTET 3.5.6 
 parameter files.  These files contain soil-based and other model
 parameter information that can be used in SAC-HTET model runs.
 Most parameter files will come in the HRAP domain and XMRG-binary format
 found commonly in NOAA NWS/OHD/RFC applications.
 For the soil parameters, LZ indicates ``lower zone'' and UZ refers to ``upper zone''.

 \var{SACHTET soiltype parameter table:} specifies the 
 dominant soiltype parameter table file.

 \var{SACHTET vegetation parameter table:} specifies the
 vegetation parameter table file. 

 \var{SACHTET parameter spatial transform:} specifies generally the SAC-HTET
  grid spatial transform.  Current option is ``none'', and future options
  will be supported.

 \var{SACHTET parameter fill option:} specifies generally the SAC-HTET
  parameter fill option.  This option is not currently supported but
 can be in the future.

 \var{SACHTET parameter fill radius:} specifies the radius with which
 to search for nearby value(s) to help fill in the missing value.

 \var{SACHTET parameter fill value:} indicates which SACHTET parameter 
 value to be used if an arbitrary value fill is needed. 

 \var{SACHTET map projection:} specifies the general SAC-HTET parameter
  grid projection.  Currently, ``hrap'' is supported and soon other projections,
  like ``latlon'' will be.

 \var{SACHTET LZFPM map:} specifies the
 lower zone primary free water (slow) maximum storage [mm]

 \var{SACHTET LZFSM map:} specifies the
 lower zone supplemental free water (fast) maximum storage [mm]

 \var{SACHTET LZPK map:} specifies the
 lower zone primary free water depletion rate [day$^{-1}$]

 \var{SACHTET LZSK map:} specifies the
 lower zone supplemental free water depletion rate [day$^{-1}$]

 \var{SACHTET LZTWM map:} specifies the
 lower zone tension water maximum storage [mm]

 \var{SACHTET UZFWM map:} specifies the
 upper zone free water maximum storage [mm]

 \var{SACHTET UZTWM map:} specifies the
 upper zone tension water maximum storage [mm]

 \var{SACHTET UZK map:} specifies the
 upper zone free water latent depletion rate [day$^{-1}$]

 \var{SACHTET PFREE map:} specifies the
 fraction percolation from upper to lower free water storage [day$^{-1}$]

 \var{SACHTET REXP map:} specifies the
 exponent of the percolation equation (percolation parameter) [-]

 \var{SACHTET ZPERC map:} specifies the
 maximum percolation rate [-]

 \var{SACHTET EFC map:} specifies the
 fraction of forest cover [-]

 \var{SACHTET PCTIM map:} specifies the
 impervious fraction of the watershad area [-]

 \var{SACHTET ADIMP map:} specifies the
 additional impervious area [-]

 \var{SACHTET SIDE map:} specifies the
 ratio of deep recharge to channel base flow [-]

 \var{SACHTET RIVA map:} specifies the
 riparian vegetation area [-]

 \var{SACHTET RSERV map:} specifies the
 fraction of lower zone free water not transferable to tension water [-]

 \var{SACHTET TBOT map:} specifies the
 bottom boundary soil temperature [C]

 \var{SACHTET STXT map:} specifies the SAC-HTET domain soil texture map file.

 \var{SACHTET CKSL map:} specifies the
 ratio of frozen to non-frozen surface (increase in frozen ground contact, usually = 8 s/m) [s/m]

 \var{SACHTET RSMAX map:} specifies the
 maximum residual porosity (usually = 0.58) [-]

 \var{SACHTET ZBOT map:} specifies the
 lower boundary depth (negative value, usually = -2.5 m) [m]

 \var{SACHTET offset time map:} specifies the path to the
  time offset map.

 \var{SACHTET soil albedo map:} specifies the soil albed map.

 

 \begin{Verbatim}[frame=single]
SACHTET soiltype parameter table:    ./testcase/sachtet_soilparms.txt
SACHTET vegetation parameter table:  ./testcase/sachtet_vegparms.txt
SACHTET LZFPM map:          ./testcase/sac_LZFPM.gz
SACHTET LZFSM map:          ./testcase/sac_LZFSM.gz
SACHTET LZPK map:           ./testcase/sac_LZPK.gz
SACHTET LZSK map:           ./testcase/sac_LZSK.gz
SACHTET LZTWM map:          ./testcase/sac_LZTWM.gz
SACHTET UZFWM map:          ./testcase/sac_UZFWM.gz
SACHTET UZTWM map:          ./testcase/sac_UZTWM.gz
SACHTET UZK map:            ./testcase/sac_UZK.gz
SACHTET PFREE map:          ./testcase/sac_PFREE.gz
SACHTET REXP map:           ./testcase/sac_REXP.gz
SACHTET ZPERC map:          ./testcase/sac_ZPERC.gz
SACHTET EFC map:            ./testcase/sac_EFC.gz
SACHTET PCTIM map:          ./testcase/sac_PCTIM.gz
SACHTET soil albedo map:    ./testcase/sachtet_soilAlbedo.gz
SACHTET offset time map:    ./testcase/sachtet_offsetTime.gz
SACHTET STXT map:           ./testcase/frz_STXT.gz
SACHTET TBOT map:           ./testcase/frz_TBOT.gz
SACHTET CKSL map:                     none
SACHTET RSMAX map:                    none
SACHTET ZBOT map:                     none
SACHTET parameter spatial transform:  none
SACHTET parameter fill option:        none
SACHTET parameter fill radius:
SACHTET parameter fill value:
SACHTET map projection:               hrap
SACHTET offset time map:
 \end{Verbatim}

 
 \var{SNOW17 parameter files:} specifies the locations of SNOW-17 
 parameter files.  These files contain snow and soil-based
 parameter information that can be used in the SNOW-17 model run.

 \var{SNOW17 ADC directory:} specifies the location of the
 multiband Snow-17 curve coordinates.

 \var{SNOW17 ADC number of points:} specifies the number
  of areal depletion curve (ADC) points along the curve defining snow depletion
  rates.

 \var{SNOW17 PGM map:} specifies the ground melt (in mm) input map.

 \var{SNOW17 parameter spatial transform:} specifies the general grid spatial
  transform option for SNOW-17.  Only current option for now is ``none''.

 \var{SNOW17 parameter fill option:} specifies the general SNOW-17
 parameter fill option.  This option is not currently supported
 but can be in the future.

 \var{SNOW17 parameter fill radius:} specifies the radius with which
 to search for nearby value(s) to help fill in the missing value.

 \var{SNOW17 parameter fill value:} indicates which  SNOW17 parameter
 value to be used if an arbitrary value fill is needed. 

 \var{SNOW17 map projection:} specfies the general SNOW-17 parameter
  map projection.  Currently only ``hrap'' is supported.  Others like, ``latlon'',
  will be supported in the future.

 \var{SNOW17 MFMAX map:} specifies the
 maximum melt factor [mm/(6hrC)]

 \var{SNOW17 MFMIN map:} specifies the
 minimum melt factor [mm/(6hrC)]

 \var{SNOW17 UADJ map:} specifies the
 the average wind function during rain-on-snow periods [mm/mb]

 \var{SNOW17 ALAT map:} specifies the
 latitude [-]

 \var{SNOW17 ELEV map:} specifies the
 elevation [m]

 \var{SNOW17 SCF map:} specifies the
 snow fall correction factor [-]

 \var{SNOW17 NMF map:} specifies the
 maximum negative melt factor [mm/(6hrC)]

 \var{SNOW17 SI map:} specifies the
 areal water-equivalent above which 100 percent areal snow cover  [mm]

 \var{SNOW17 MBASE map:} specifies the
 base temperature for non-rain melt factor [C]

 \var{SNOW17 PXTEMP map:} specifies the
 temperature which spereates rain from snow  [C]

 \var{SNOW17 PLWHC map:} specifies the
 maximum amount of liquid-water held against gravity drainage  [-]

 \var{SNOW17 TIPM map:} specifies the
 antecedent snow temperature index parameter [-]

 \var{SNOW17 LAEC map:} specifies the
 snow-rain split temperature [C]

 
 \begin{Verbatim}[frame=single]
SNOW17 MFMAX map:       ./testcase/snow_MFMAX.gz
SNOW17 MFMIN map:       ./testcase/snow_MFMIN.gz
SNOW17 UADJ map:        ./testcase/snow_UADJ.gz
SNOW17 ALAT map:        ./testcase/snow_ALAT.gz
SNOW17 ELEV map:        ./testcase/snow_ELEV.gz
SNOW17 SCF map:                 none
SNOW17 NMF map:                 none
SNOW17 SI map:                  none
SNOW17 MBASE map:               none
SNOW17 PXTEMP map:              none
SNOW17 PLWHC map:               none
SNOW17 TIPM map:                none
SNOW17 PGM map:                 none
SNOW17 ELEV map:                none
SNOW17 LAEC map:                none
SNOW17 ADC directory:           none
SNOW17 ADC number of points:     11
SNOW17 parameter spatial transform:  none
SNOW17 parameter fill option:        none
SNOW17 parameter fill radius:
SNOW17 parameter fill value:
SNOW17 map projection:               hrap
 \end{Verbatim}


 
 \subsubsection{WRSI model parameter files} 
 \var{WRSI landmask file:} specifies the location of the GeoWRSI 2.0
   land mask file (default file is in *BIL format).

 \var{WRSI length of growing period file:} specifies the 
   location of the GeoWRSI 2.0 length of growing period file
   (default file is in *BIL format).

 \var{WRSI water holding capacity file:} specifies the 
   location of the GeoWRSI 2.0 water holding capacity file
   (default file is in *BIL format).

 \var{WRSI WRSI climatology file:} specifies the 
   location of the GeoWRSI 2.0 WRSI climatology file
   (default file is in *BIL format).

 \var{WRSI SOS climatology file:} specifies the 
   location of the GeoWRSI 2.0 SOS climatology file
   (default file is in *BIL format).

 \var{WRSI SOS file:} specifies the 
   location of an (optional) current start-of-season (SOS) file
   (default file is in *BIL format).

 \var{WRSI SOS anomaly file:} specifies the 
   location of an (optional) current (SOS) anomaly file
   (default file is in *BIL format).
 

 \begin{Verbatim}[frame=single]
WRSI landmask file:                 ./data/Africa/Static/sawmask
WRSI length of growing period file: ./data/Africa/Static/lgp_south
WRSI water holding capacity file:   ./data/Africa/Static/whc3
WRSI WRSI climatology file:         ./data/Africa/Static/wsimedn_edc_s
WRSI SOS climatology file:          ./data/Africa/SOS/sosmedn_edc_s
WRSI SOS file:                       none
WRSI SOS anomaly file:               none
 \end{Verbatim}

 
 \subsection{Climate Parameters} \label{ssec:climparams}
 

 
 Climatology parameter maps

 \var{PPT climatology data source:} specifies the monthly
 precipitation (PPT) climatology fields.
 Current source options include:

 \begin{tabular}{ll}
 Value       & Description               \\
 PRISM       & PRISM US-only climate downscaled fields. For more info, \\
             &  see:  http://www.prism.oregonstate.edu/   \\
 WORLDCLIM   & Global climate layers downscaled.  For more info, see:  \\
             &  http://www.worldclim.org/ \\
 \end{tabular}

 \var{PPT climatology maps:} specifies the source of the climatology
 based precipitation files.  The climatology precipitation data
 files can have the following naming conventions, depending
 on the data source: \\

 PRISM: $<$directory$>$/$<$file header$>$.$<$tag$>$.txt
 \begin{itemize} 
 \item[] The file header can be anything (such as ppt\_1931\_2010).
 \item[] The tag should represent the month
         (such as jan, feb, mar, etc.).
 \end{itemize} 
 WORLDCLIM:  $<$directory$>$/$<$file header$>$.$<$tag$>$.1gd4r
 \begin{itemize} 
 \item[] The file header can be prec\_
 \item[] The tag should represent the month (such as 1, 2,..., 12).
 \end{itemize} 
 \var{PPT climatology interval:} specifies the frequency of the
 precipitation climatology in months.  Current option is: ``monthly``.
 

 \begin{Verbatim}[frame=single]
PPT climatology data source:  PRISM
PPT climatology maps:  ../LS_PARAMETERS/climate_maps/ppt_1981_2010
PPT climatology interval:     monthly
 \end{Verbatim}


 
 \var{Climate params spatial transform:} indicates which spatial
 transform (i.e., upscale or downscale) type is to be applied to
 climate parameters. Only ``average'' spatial transform works currently
 for the ``WORLDCLIM'' climatology files.
 Options include:

 \begin{tabular}{ll}
 Value   & Description                                   \\
 none    & Data is on same grid as LIS output domain     \\
 average   & Upscale by averaging values for each gridcell       \\
 neighbor  & Reinterpolate by selecting nearest gridcell neighbor \\
 bilinear  & Reinterpolate by using bilinear interpolation        \\
 budget-bilinear & Reinterpolate by using conservative, budget-bilinear \\
 \end{tabular}
 

 \begin{Verbatim}[frame=single]
Climate params spatial transform:   average
 \end{Verbatim}

 
 This section also outlines the domain specifications of
 climatology-based parameters, like higher scaled monthly
 precipitation or min/max temperatures.
 If the map projection of parameter data is specified to be lat/lon,
 the following configuration should be used for specifying
 climatology data.
 See Appendix~\ref{sec:d_latlon_example} for more details about
 setting these values.
 

 \begin{Verbatim}[frame=single]
Climate params map projection:     latlon
 \end{Verbatim}



 
 \subsection{Forcing Parameters} \label{ssec:forcings}
 

 
 \subsubsection{NLDAS-2 Forcing based parameter inputs}

 \var{NLDAS2 elevation difference map:} specifies the NLDAS-2 elevation 
 difference file used to remove built-in elevation correction.  

 \var{NARR terrain height map:} specifies the terrain height map
 for the NLDAS-2 base forcing of the North American Regional
 Reanalysis (NARR).

 If the run mode option selected is ``Metforce processing'' or
 ``Metforce temporal downscaling'', please see the latest LIS Users'
 Guide.

 

 \begin{Verbatim}[frame=single]
NLDAS2 elevation difference map: ../NARR_elev-diff.1gd4r
NARR terrain height map:         ../NARR_elevation.1gd4r
 \end{Verbatim}

 
 \subsubsection{NLDAS-1 Forcing based parameter inputs}

 \var{NLDAS1 elevation difference map:} specifies the NLDAS-1 elevation 
 difference file used to remove built-in elevation correction.  

 \var{EDAS terrain height map:} specifies the terrain height map
 for the NLDAS-1 base forcing of the Eta Data Assimilation
 System (EDAS).

 If the run mode option selected is ``Metforce processing'' or
 ``Metforce temporal downscaling'', please see the latest LIS Users'
 Guide.

 

 \begin{Verbatim}[frame=single]
NLDAS1 elevation difference map: ../NLDAS1/EDAS_elev-diff.1gd4r
EDAS terrain height map:         ../NLDAS1/EDAS_elevation.1gd4r
 \end{Verbatim}

 
 \subsubsection{PRINCETON Forcing based parameter inputs}

 \var{PRINCETON elevation map:} specifies the terrain height map
 for the Princeton University global forcing dataset.

 If the run mode option selected is ``Metforce processing'' or
 ``Metforce temporal downscaling'', please see the latest LIS Users'
 Guide.

 

 \begin{Verbatim}[frame=single]
PRINCETON elevation map:     ../PRINCETON/hydro1k_elev_mean_1d.asc
 \end{Verbatim}

 
 \subsubsection{NAM242 Forcing based parameter inputs}

 \var{NAM242 elevation map:} specifies the terrain height map
 for the North American Mesoscale (NAM) NOAA grid 242 forcing dataset.

 If the run mode option selected is ``Metforce processing'' or
 ``Metforce temporal downscaling'', please see the latest LIS Users'
 Guide.

 

 \begin{Verbatim}[frame=single]
NAM242 elevation map:     ../NAM/terrain.242.grb
 \end{Verbatim}

 
 \subsubsection{GDAS} \label{sssec:forcings_gdas}
 GDAS parameter inputs:  GDAS elevation maps specify lowest
 boundary layer information which can be used to downscale or
 lapse rate adjust GDAS meteorological variables, if given a higher 
 resolution elevation height map. Original files are given in
 Grib-1 format and on their original Gaussian grids (from NCEP), 
 so the GDAS elevation file reader is set up to support these files.

 \var{GDAS forcing directory:} specifies the location of the GDAS
 forcing data files.

 \var{GDAS T126 elevation map:} specifies the GDAS T126 elevation
 definition.

 \var{GDAS T170 elevation map:} specifies the GDAS T170 elevation
 definition.

 \var{GDAS T254 elevation map:} specifies the GDAS T254 elevation
 definition.

 \var{GDAS T382 elevation map:} specifies the GDAS T382 elevation
 definition.

 \var{GDAS T574 elevation map:} specifies the GDAS T574 elevation
 definition.

 \var{GDAS T1534 elevation map:} specifies the GDAS T1534 elevation
 definition.

 If the run mode option selected is ``Metforce processing'' or
 ``Metforce temporal downscaling'', please see the latest LIS Users'
 Guide.

 

 \begin{Verbatim}[frame=single]
GDAS forcing directory:
GDAS T126 elevation map:  ./GDAS/global_orography.t126.grb
GDAS T170 elevation map:  ./GDAS/global_orography.t170.grb
GDAS T254 elevation map:  ./GDAS/global_orography.t254.grb
GDAS T382 elevation map:  ./GDAS/global_orography.t382.grb
GDAS T574 elevation map:  ./GDAS/global_orography.t574.grb
GDAS T1534 elevation map: ./GDAS/global_orography_uf.t1534.3072.1536.grb
 \end{Verbatim}

 
 \subsubsection{ECMWF} \label{sssec:forcings_ecmwf}
 ECMWF parameter inputs:  ECMWF elevation maps specify lowest
 boundary layer information which can be used to downscale or
 lapse rate adjust ECMWF meteorological variables, if given a higher 
 resolution elevation height map. Original files are given in
 Grib-1 format and on their original lat-lon grids (from ECMWF), 
 so the ECMWF elevation file reader is set up to support these files.

 \var{ECMWF forcing directory:} specifies the location of the
 ECMWF forcing data files.

 \var{ECMWF IFS23R4 elevation map:} specifies the ECMWF IFS23R4 
 terrain height map file path.

 \var{ECMWF IFS25R1 elevation map:} specifies the ECMWF IFS25R1
 terrain height map file path.

 \var{ECMWF IFS30R1 elevation map:} specifies the ECMWF IFS30R1 
 terrain height map file path.

 \var{ECMWF IFS33R1 elevation map:} specifies the ECMWF IFS33R1 
 terrain height map file path.

 \var{ECMWF IFS35R2 elevation map:} specifies the ECMWF IFS35R2
 terrain height map file path.

 \var{ECMWF IFS35R3 elevation map:} specifies the ECMWF IFS35R3
 terrain height map file path.

 \var{ECMWF IFS36R1 elevation map:} specifies the ECMWF IFS36R1
 terrain height map file path.

 \var{ECMWF IFS37R2 elevation map:} specifies the ECMWF IFS37R2
 terrain height map file path.


 If the run mode option selected is ``Metforce processing'' or
 ``Metforce temporal downscaling'', please see the latest LIS Users'
 Guide.

 

 \begin{Verbatim}[frame=single]
ECMWF forcing directory:
ECMWF IFS23R4 elevation map:  ./ECMWF/ecmwf.2001092006.092006.elev_1_4
ECMWF IFS25R1 elevation map:  ./ECMWF/ecmwf.2003010806.010806.elev_1_4
ECMWF IFS30R1 elevation map:  ./ECMWF/ecmwf.2006020106.020106.elev_1_4
ECMWF IFS33R1 elevation map:  ./ECMWF/ecmwf.2008060306.060306.elev_1_4
ECMWF IFS35R2 elevation map:  ./ECMWF/ecmwf.2009031006.031006.elev_1_4
ECMWF IFS35R3 elevation map:  ./ECMWF/ecmwf.2009090806.090806.elev_1_4
ECMWF IFS36R1 elevation map:  ./ECMWF/ecmwf.2010012606.012606.elev_1_4
ECMWF IFS37R2 elevation map:  ./ECMWF/ecmwf.2011051806.051806.elev_1_4
 \end{Verbatim}

 
 \subsubsection{ECMWF Reanalysis Forcing based parameter inputs}

 \var{ECMWF Reanalysis forcing directory:} specifies the
 location of the ECMWF Reanalysis forcing data files.

 \var{ECMWF Reanalysis maskfile:} specifies the ECMWF Reanalysis
 mask file.

 \var{ECMWF Reanalysis elevation map:} specifies the ECMWF Reanalysis  
 elevation file.  

 \var{ECMWF Reanalysis elevation spatial transform:} specifies 
  the terrain height map spatial grid transform option (e.g., average).

 If the run mode option selected is ``Metforce processing'' or
 ``Metforce temporal downscaling'', please see the latest LIS Users'
 Guide.

 

 \begin{Verbatim}[frame=single]
ECMWF Reanalysis forcing directory:
ECMWF Reanalysis elevation map: ./metforcing_parms/ECMWFRean/elev_ECMWF-reanalysis.1gd4r
ECMWF Reanalysis elevation spatial transform:   "average"
ECMWF Reanalysis maskfile:
 \end{Verbatim}


 
 \subsubsection{MERRA-2 Forcing based parameter inputs}

 \var{MERRA2 geopotential terrain height file:} specifies 
  the MERRA-2 geopotential height file, which gets
  converted to terrain height (in meters) in LDT.

 

 \begin{Verbatim}[frame=single]
MERRA2 geopotential terrain height file: ./MERRA2_100/MERRA2_101.const_2d_asm_Nx.00000000.nc4
 \end{Verbatim}


 
 \subsubsection{TRMM 3B42RTV7 precipitation}

 If the run mode option selected is ``Metforce processing'' or
 ``Metforce temporal downscaling'', please see the latest LIS Users'
 Guide.

 

 
 \subsubsection{TRMM 3B42V6 precipitation} 

 If the run mode option selected is ``Metforce processing'' or
 ``Metforce temporal downscaling'', please see the latest LIS Users'
 Guide.

 

 
 \subsubsection{TRMM 3B42V7 precipitation} 

 If the run mode option selected is ``Metforce processing'' or
 ``Metforce temporal downscaling'', please see the latest LIS Users'
 Guide.

 

 
 \subsubsection{CMAP precipitation}

 If the run mode option selected is ``Metforce processing'' or
 ``Metforce temporal downscaling'', please see the latest LIS Users'
 Guide.

 

 
 \subsubsection{CMORPH precipitation} 

 If the run mode option selected is ``Metforce processing'' or
 ``Metforce temporal downscaling'', please see the latest LIS Users'
 Guide.

 

 
 \subsubsection{MERRA-Land forcing}

 If the run mode option selected is ``Metforce processing'' or
 ``Metforce temporal downscaling'', please see the latest LIS Users'
 Guide.

 

 
 \subsubsection{MERRA2 forcing}

 If the run mode option selected is ``Metforce processing'' or
 ``Metforce temporal downscaling'', please see the latest LIS Users'
 Guide.

 

 
 \subsubsection{RDHM356 forcing} 

 If the run mode option selected is ``Metforce processing'' or
 ``Metforce temporal downscaling'', please see the latest LIS Users'
 Guide.

 

 
 \subsubsection{RFE2Daily precipitation} 

 If the run mode option selected is ``Metforce processing'' or
 ``Metforce temporal downscaling'', please see the latest LIS Users'
 Guide.

 

 
 \subsubsection{RFE2gdas precipitation} 

 If the run mode option selected is ``Metforce processing'' or
 ``Metforce temporal downscaling'', please see the latest LIS Users'
 Guide.

 

 
 \subsubsection{CHIRPSv2 precipitation} 

 If the run mode option selected is ``Metforce processing'' or
 ``Metforce temporal downscaling'', please see the latest LIS Users'
 Guide.

 

 
 \subsubsection{Stage II precipitation}

 If the run mode option selected is ``Metforce processing'' or
 ``Metforce temporal downscaling'', please see the latest LIS Users'
 Guide.

 

 
 \subsubsection{Stage IV precipitation} 

 If the run mode option selected is ``Metforce processing'' or
 ``Metforce temporal downscaling'', please see the latest LIS Users'
 Guide.

 


 
 \subsubsection{GEOS5 forecast} 

 If the run mode option selected is ``Metforce processing'' or
 ``Metforce temporal downscaling'', please see the latest LIS Users'
 Guide.

 

 
 \subsubsection{GFS} 

 If the run mode option selected is ``Metforce processing'' or
 ``Metforce temporal downscaling'', please see the latest LIS Users'
 Guide.

 


 
 \subsection{LIS restart preprocessing options} \label{ssec:rstopts}
 

 
 \var{Input restart file directory:} \attention{specifies what?}

 \var{Input restart file naming style:} \attention{specifies what?}

 \var{Input restart file output interval:} \attention{specifies what?}

 \var{Input restart model timestep used:} \attention{specifies what?}

 \var{Output restart file generation mode:} \attention{specifies what?}

 \var{Output restart file averaging interval type:} \attention{specifies what?}
 

 \begin{Verbatim}[frame=single]
Input restart file directory:
Input restart file naming style:
Input restart file output interval:
Input restart model timestep used:
Output restart file generation mode:
Output restart file averaging interval type:
 \end{Verbatim}


 
 \subsection{Ensemble restart model options} \label{ssec:ensrstopts}
 


 
 \var{LIS restart source:} 
 specifies the surface model restart file source. Options are:

 \begin{tabular}{ll}
 Value     & Description                   \\
 LSM       & LSM restart file type         \\ 
 Routing   & river or streamflow routing 
             model restart file type       \\
 \end{tabular}
 

 \begin{Verbatim}[frame=single]
LIS restart source:   "LSM"
 \end{Verbatim}

 
 \var{Ensemble restart generation mode:} 
 specifies the mode of ensemble restart generation. Options are:

 \begin{tabular}{ll}
 Value     & Description                          \\
 upscale   & convert from a single member restart
             to a multi-member restart            \\
 downscale & convert from a multi-member restart
             to a single member restart           \\
 \end{tabular}
 

 \begin{Verbatim}[frame=single]
Ensemble restart generation mode:   "upscale"
 \end{Verbatim}

 
 \var{Input restart filename:} 
 specifies the name of the input restart file. 
 

 \begin{Verbatim}[frame=single]
Input restart filename: ../OL/LIS_RST_NOAH33_201001010000.d01.nc
 \end{Verbatim}

 
 \var{Output restart filename:} 
 specifies the name of the output restart file. 
 

 \begin{Verbatim}[frame=single]
Output restart filename: ./LIS_RST_NOAH33_201001010000.d01.nc
 \end{Verbatim}

 
 \var{Number of ensembles per tile (input restart):} 
 specifies the number of ensemble members used in the input restart
 file. 
 

 \begin{Verbatim}[frame=single]
Number of ensembles per tile (input restart): 1
 \end{Verbatim}

 
 \var{Number of ensembles per tile (output restart):} 
 specifies the number of ensemble members to be used in the output
 restart file. 
 

 \begin{Verbatim}[frame=single]
Number of ensembles per tile (output restart): 12
 \end{Verbatim}

 
 \var{Note: Make sure to specify the surface type, veg, soil, etc., subgrid tiling entries} 
 For upscaling or downscaling of restart files, maximum number of tiles and minimum 
 cutoff percentage entries for subgrid tiling based on vegetation or other 
 parameter types (e.g., soil type, elevation, etc.) are required as entries.

 For example, must include, Maximum number of surface type tiles per grid:
 


 
 \subsection{NUWRF preprocessing for real options} \label{ssec:nuwrfrealopts}
 

 
 The section describes some of the LDT-based NUWRF real input processing
 options.

 \var{LIS history file for land state initialization:} specifies the 
 file name of the LIS history file to use to initialize the land state.

 \var{Processed NUWRF file for input to real:} specifies the file
 name of the generated file that is then used as input to the real.exe
 program in NUWRF.

 
 \begin{Verbatim}[frame=single]
LIS history file for land state initialization:   EXAMPLE
Processed NUWRF file for input to real:     EXAMPLE
 \end{Verbatim}


 
 \subsection{Data Assimilation preprocessing options}
 \label{ssec:dapreprocopts}
 


 
 The start time is specified in the following format:

 \begin{tabular}{lll}
 Variable & Value & Description                      \\
 \var{Starting year:} & integer 2001 -- present &
                        specifying starting year     \\
 \var{Starting month:} & integer 1 -- 12 &
                        specifying starting month    \\
 \var{Starting day:} & integer 1 -- 31 &
                       specifying starting day       \\
 \var{Starting hour:} & integer 0 -- 23 &
                        specifying starting hour     \\
 \var{Starting minute:} & integer 0 -- 59 &
                          specifying starting minute \\
 \var{Starting second:} & integer 0 -- 59 &
                          specifying starting second \\
 \end{tabular}
 \nextpar
 

 \begin{Verbatim}[frame=single]
Starting year:                2002
Starting month:                1
Starting day:                  2
Starting hour:                 0
Starting minute:               0
Starting second:               0
 \end{Verbatim}

 
 The end time is specified in the following format:

 \begin{tabular}{lll}
 Variable & Value & Description                    \\
 \var{Ending year:} & integer 2001 -- present &
                        specifying ending year     \\
 \var{Ending month:} & integer 1 -- 12 &
                        specifying ending month    \\
 \var{Ending day:} & integer 1 -- 31 &
                       specifying ending day       \\
 \var{Ending hour:} & integer 0 -- 23 &
                        specifying ending hour     \\
 \var{Ending minute:} & integer 0 -- 59 &
                          specifying ending minute \\
 \var{Ending second:} & integer 0 -- 59 &
                          specifying ending second \\
 \end{tabular}
 

 \begin{Verbatim}[frame=single]
Ending year:                  2010
Ending month:                  1
Ending day:                    1
Ending hour:                   0
Ending minute:                 0
Ending second:                 0
 \end{Verbatim}

 
 \var{LIS output timestep:} specifies the LIS output time-step.
 

 \begin{Verbatim}[frame=single]
LIS output timestep:          1da
 \end{Verbatim}

 
 \var{DA observation source:} 
 specifies the source of the observation data on which preprocessing is
 performed.  Options are:

 \begin{tabular}{ll}
 Value                      & Description                           \\
 LIS LSM soil moisture      & soil moisture output from a LIS run   \\
 Synthetic soil moisture    & synethtic soil moisture observations  \\
                            & created from a LIS run                \\
 AMSR-E(LPRM) soil moisture & Land Parameter Retrieval Model (LPRM) \\
                            & retrievals of AMSR-E soil moisture    \\
 AMSR-E(NASA) soil moisture & NASA AMSR-E soil moisture             \\
 ESA CCI soil moisture      & Essential Climate Variable (ECV) soil \\
                            & moisture retrievals                   \\
 WindSat soil moisture      & WindSat retrievals of soil moisture   \\
 RT SMOPS soil moisture     & Real Time Soil Moisture Operational   \\
                            & Processing System (SMOPS) based soil  \\
                            & moisture retrievals                   \\
 ASCAT TUW soil moisture    & ASCAT soil moisture retrievals from
                              TU Wein                               \\
 GRACE TWS                  & Terrestrial water storage observations
                              from GRACE                            \\ 
 Simulated GRACE            & Simulated water storage observations
                              from GRACE                            \\ 
 \end{tabular} 
 

 \begin{Verbatim}[frame=single]
DA observation source:   "AMSR-E(LPRM) soil moisture"
 \end{Verbatim}

 
 \var{DA preprocessing method:}
 specifies which preprocessing method should be used
 Acceptable values are:

 \begin{tabular}{ll}
 Value     & Description                                  \\
 "Obs grid generation"  & Create the observation space grid only \\
 "CDF generation"       & Create CDFs for the given data source \\
 "Anomaly correction"   & Create updated observations for DA by \\
                        & applying anomaly correction (Used for GRACE DA)\\
 \end{tabular}

 

 \begin{Verbatim}[frame=single]
DA preprocessing method:  "CDF generation"
 \end{Verbatim}

 
 \var{Name of the preprocessed DA file:}
 specifies the name of the preprocessed DA file from LDT. 
 

 \begin{Verbatim}[frame=single]
Name of the preprocessed DA file: "lprm_cdf"
 \end{Verbatim}

 
 \var{Number of bins to use in the CDF:}
 specifies the number of bins to use while computing the CDF.
 

 \begin{Verbatim}[frame=single]
Number of bins to use in the CDF: 100 
 \end{Verbatim}

 
 \var{Temporal resolution of CDFs:} specifies whether to generate
 lumped (considering all years and all seasons) CDFs or to stratify CDFs
 for each calendar month.
 Acceptable values are:

 \begin{tabular}{ll}
 Value     & Description                                  \\
 monthly   & stratify for each calendar month             \\
 yearly    & lump (considering all years and all seasons) \\
 \end{tabular}
 

 \begin{Verbatim}[frame=single]
Temporal resolution of CDFs:     monthly
 \end{Verbatim}

 
 \var{Enable spatial sampling for CDF calculations:}
 Normally CDFs are calculated (for a given grid cell) by using 
 the data values available at that grid point only. If this option
 is enabled, then values around a specified radius will be used
 in the CDF calculations, effectively improving the sampling
 density at the risk of reduced geographic specificity. 
 

 \begin{Verbatim}[frame=single]
Enable spatial sampling for CDF calculations: 1
 \end{Verbatim}

 
 \var{Spatial sampling window radius for CDF calculations:}
 specifies the radius with which
 to search for nearby value(s) in the CDF calculations. 
 

 \begin{Verbatim}[frame=single]
Spatial sampling window radius for CDF calculations: 2
 \end{Verbatim}

 
 \var{Group CDFs by external data:} specifies whether to group CDFs
 for each pixel by an externally specified categorical map; for example,
 by landcover.
 Acceptable values are:

 \begin{tabular}{ll}
 Value & Description                   \\
 0     & do not group by external data \\
 1     & group by external data        \\
 \end{tabular}
 

 \begin{Verbatim}[frame=single]
Group CDFs by external data:     0
 \end{Verbatim}

 
 \var{CDF grouping attributes file:} specifies the name of an
 ASCII file that specifies the attributes of the CDF grouping,
 if enabled.  A sample file is shown below.
 The first line is a description.  The second line is
 the name of the file containing the external data for grouping.
 The third line is a descrition.  The fourth line is the minimum
 value of the categorical data, followed by the maximum value of the
 categorical data, followed by the number of bins of the categorical
 data.

 \#category file  \\
 landcover.1gd4r  \\
 \#min max nbins  \\
 1 19 18          \\
 

 \begin{Verbatim}[frame=single]
CDF grouping attributes file:     cdf_grouping.txt
 \end{Verbatim}

 
 \var{Temporal averaging interval:}
 specifies temporal averaging interval to be used while computing
 the CDF.
 

 \begin{Verbatim}[frame=single]
Temporal averaging interval:  "1da"
 \end{Verbatim}

 
 \var{Apply external mask:}
 specifies if an external mask (time varying) is to be applied
 while computing the CDF.
 

 \begin{Verbatim}[frame=single]
Apply external mask:   0 
 \end{Verbatim}

 
 \var{External mask directory:}
 specifies the location of the external mask.
 

 \begin{Verbatim}[frame=single]
External mask directory:  none
 \end{Verbatim}


 
 \var{Observation count threshold:}
 specifies the minimum number of observations to be used for generating
 valid CDF data.
 

 \begin{Verbatim}[frame=single]
Observation count threshold:  500
 \end{Verbatim}


 
 \var{LIS soil moisture output format:}
 specifies the output format of the LIS model output.
 (binary/netcdf/grib1)

 \var{LIS soil moisture output methodology:}
 specifies the output methodology used in the LIS model output
 (1d tilespace/1d gridspace/2d gridspace).

 \var{LIS soil moisture output naming style:}
 specifies the output naming style used in the LIS model output 
 (3 level hierarchy/4 level hierarchy, etc.).

 \var{LIS soil moisture output nest index:}
 specifies the index of the nest used in the LIS model output.

 \var{LIS soil moisture output directory:}
 specifies the location of the LIS model output.

 \var{LIS soil moisture output timestep:} specifies the output
 timestep of the LIS soil moisture.

 \var{LIS soil moisture output map projection:} 
 specifies the map projection used in the LIS model output.

 For Lat/Lon projections:

 \var{LIS soil moisture domain lower left lat:}
 specifies the lower left latitude of the LIS model output.

 \var{LIS soil moisture domain lower left lon:}
 specifies the lower left longitude of the LIS model output.

 \var{LIS soil moisture domain upper right lat:}
 specifies the upper right latitude of the LIS model output.

 \var{LIS soil moisture domain upper right lon:}
 specifies the upper right longitude of the LIS model output.

 \var{LIS soil moisture domain resolution (dx):}
 specifies the resolution (in degrees) along the latitude of the
 LIS model output.

 \var{LIS soil moisture domain resolution (dy):}
 specifies the resolution (in degrees) along the longitude of the
 LIS model output.

 For Lambert and polar projections:

 \var{LIS soil moisture domain lower left lat:}
 specifies the lower left latitude of the LIS model output

 \var{LIS soil moisture domain lower left lon:}
 specifies the lower left longitude of the LIS model output

 \var{LIS soil moisture domain true lat1:}
 specifies the true lat1 of the LIS model output

 \var{LIS soil moisture domain true lat2:}
 specifies the true lat2 of the LIS model output

 \var{LIS soil moisture domain standard lon:}
 specifies the standard longitude of the LIS model output

 \var{LIS soil moisture domain resolution:}
 specifies the resolution of the LIS model output

 \var{LIS soil moisture domain x-dimension size:}
 specifies the x-dimension size of the LIS model output

 \var{LIS soil moisture domain y-dimension size:}
 specifies the y-dimension size of the LIS model output

 For "WMO convention" style output

 \var{LIS soil moisture security class:}
 specifies the security classification for the LIS model output file,
 used only for WMO-convention output.

 \var{LIS soil moisture distribution class:}
 specifies the distribution classification for the LIS model output file,
 used only for WMO-convention output.

 \var{LIS soil moisture data category:}
 specifies the data category for the LIS model output file,
 used only for WMO-convention output.

 \var{LIS soil moisture area of data:}
 specifies the area of data for the LIS model output file,
 used only for WMO-convention output.

 \var{LIS soil moisture write interval:}
 specifies the write interval for the LIS model output file,
 used only for WMO-convention output.
 

 \begin{Verbatim}[frame=single]
LIS soil moisture output format:            "netcdf"
LIS soil moisture output methodology:       "2d gridspace"
LIS soil moisture output naming style:      "3 level hierarchy"
LIS soil moisture output map projection:    "latlon"
LIS soil moisture output nest index:        1
LIS soil moisture output timestep:          EXAMPLE
LIS soil moisture output directory:         ../OL/OUTPUT
LIS soil moisture domain lower left lat:    18.375
LIS soil moisture domain lower left lon:    -111.375
LIS soil moisture domain upper right lat:   41.375
LIS soil moisture domain upper right lon:   -85.875
LIS soil moisture domain resolution (dx):   0.25
LIS soil moisture domain resolution (dy):   0.25
LIS soil moisture security class:
LIS soil moisture distribution class:
LIS soil moisture data category:
LIS soil moisture area of data:
LIS soil moisture write interval:
 \end{Verbatim}

 
 \var{Synthetic soil moisture observation directory:}
 specifies the location of the data directory containing the synthetic 
 soil moisture data.

 \var{Synthetic soil moisture observation timestep:} specifies
 the timestep of the synthetic soil moisture observations.
 

 \begin{Verbatim}[frame=single]
Synthetic soil moisture observation directory:  ./SYN_SM
Synthetic soil moisture observation timestep:   EXAMPLE
 \end{Verbatim}

 
 \var{AMSR-E(LPRM) soil moisture observation directory:}
 specifies the location of the data directory containing the LPRM 
 AMSR-E data.

 \var{AMSR-E(LPRM) use raw data:}
 specifies if raw data (instead of the retrievals CDF-matched to the
 GLDAS Noah climatology).
 

 \begin{Verbatim}[frame=single]
AMSR-E(LPRM) soil moisture observation directory:  ./LPRM.v5
AMSR-E(LPRM) use raw data:           1
 \end{Verbatim}

 
 \var{NASA AMSRE soil moisture observation directory:}
 specifies the location of the data directory containing the NASA 
 AMSR-E data.

 

 \begin{Verbatim}[frame=single]
NASA AMSRE soil moisture observation directory: ./NASA_AMSRE
 \end{Verbatim}

 
 \var{ESA CCI soil moisture observation directory:}
 specifies the location of the data directory containing the ESA CCI
 soil moisture data.

 \var{ESA CCI soil moisture version of data:}
 specifies the version of the ESA CCI soil moisture dataset.
 

 \begin{Verbatim}[frame=single]
ESA CCI soil moisture observation directory:  ./ECV
ESA CCI soil moisture version of data:        EXAMPLE
 \end{Verbatim}

 
 \var{GCOMW AMSR2 L3 soil moisture observation directory:}
 specifies the location of the data directory containing the 
 GCOMW AMSR v2 L3 soil moisture data.
 

 \begin{Verbatim}[frame=single]
GCOMW AMSR2 L3 soil moisture observation directory:  ./GCOMW_AMSR2
 \end{Verbatim}

 
 \var{WindSat soil moisture observation directory:}
 specifies the location of the data directory containing the WindSat
 soil moisture data.
 

 \begin{Verbatim}[frame=single]
WindSat soil moisture observation directory:  ./WindSat
 \end{Verbatim}

 
 \var{Aquarius L2 soil moisture observation directory:}
 specifies the location of the data directory containing the Aquarius
 soil moisture data.
 

 \begin{Verbatim}[frame=single]
Aquarius L2 soil moisture observation directory:  ./Aquarias_SM/
 \end{Verbatim}

 
 \var{SMOS L2 soil moisture observation directory:}
 specifies the location of the data directory containing the SMOS 
 soil moisture data.
 

 \begin{Verbatim}[frame=single]
SMOS L2 soil moisture observation directory:  ./SMOS_SM/
 \end{Verbatim}

 
 \var{RT SMOPS soil moisture observation directory:}
 specifies the location of the data directory containing the real time
 SMOPS soil moisture data.

 \var{RT SMOPS soil moisture use ASCAT data:} 
 specifies if the ASCAT layer of SMOPS is to be used.

 \var{RT SMOPS soil moisture use WindSat data:} 
 specifies if the WindSat layer of SMOPS is to be used.

 \var{RT SMOPS soil moisture use SMOS data:} 
 specifies if the SMOS layer of SMOPS is to be used.

 

 \begin{Verbatim}[frame=single]
RT SMOPS soil moisture observation directory:  ./RTSMOPS
RT SMOPS soil moisture use ASCAT data:       1
RT SMOPS soil moisture use WindSat data:     0
RT SMOPS soil moisture use SMOS data:        0
 \end{Verbatim}

 
 \var{ASCAT (TUW) soil moisture observation directory:}
 specifies the location of the data directory containing the TU Wein
 retrievals of ASCAT soil moisture data.
 

 \begin{Verbatim}[frame=single]
ASCAT (TUW) soil moisture observation directory:  ./TUW_ASCAT
 \end{Verbatim}


 
 \var{GRACE raw data filename:}
 specifies the name of the GRACE raw data.

 \var{GRACE baseline starting year:}
 specifies the baseline starting year from which to establish the TWS
 climatology.

 \var{GRACE baseline ending year:}
 specifies the baseline ending year from which to establish the TWS
 climatology.

 \var{GRACE scale factor filename:} specifies the name of the file
 containing the GRACE scale factor.  This is NetCDF file provided
 by JPL.

 \var{GRACE measurement error filename:} specifies the name of the file
 containing the GRACE measurement error.  This is a NetCDF file provided
 by JPL.

 \var{GRACE process basin averaged observations:} specifies whether to
 process basin averaged observations.  Default value is 0.
 Acceptable values are:

 \begin{tabular}{ll}
 Value & Description    \\
 0     & do not process \\
 1     & process        \\
 \end{tabular}

 \var{GRACE basin map file:} specifies the file name of the
 basin map data.

 \var{LIS TWS output format:}
 specifies the output format of the LIS model output
 (binary/netcdf/grib1).

 \var{LIS TWS output methodology:}
 specifies the output methodology used in the LIS model output
 (1d tilespace/1d gridspace/2d gridspace).

 \var{LIS TWS output naming style:}
 specifies the output naming style used in the LIS model output 
 (3 level hierarchy/4 level hierarchy, etc.).

 \var{LIS TWS output nest index:}
 specifies the index of the nest used in the LIS model output.

 \var{LIS TWS output directory:}
 specifies the location of the LIS model output.

 \var{LIS TWS output map projection:} 
 specifies the map projection used in the LIS model output.

 For lat/lon projection:

 \var{LIS TWS output domain lower left lat:}
 specifies the lower left latitude of the LIS model output
 (if map projection is latlon).

 \var{LIS TWS output domain lower left lon:}
 specifies the lower left longitude of the LIS model output
 (if map projection is latlon).

 \var{LIS TWS output domain upper right lat:}
 specifies the upper right latitude of the LIS model output
 (if map projection is latlon).

 \var{LIS TWS output domain upper right lon:}
 specifies the upper right longitude of the LIS model output
 (if map projection is latlon).

 \var{LIS TWS output domain resolution (dx):}
 specifies the resolution (in degrees) along the latitude of the
 LIS model output (if map projection is latlon).

 \var{LIS TWS output domain resolution (dy):}
 specifies the resolution (in degrees) along the longitude of the
 LIS model output (if map projection is latlon).

 For Lambert and polar projections:

 \var{LIS TWS output domain lower left lat:}
 specifies the lower left latitude of the LIS model output

 \var{LIS TWS output domain lower left lon:}
 specifies the lower left longitude of the LIS model output

 \var{LIS TWS output domain true lat1:}
 specifies the true lat1 of the LIS model output

 \var{LIS TWS output domain true lat2:}
 specifies the true lat2 of the LIS model output

 \var{LIS TWS output domain standard lon:}
 specifies the standard longitude of the LIS model output

 \var{LIS TWS output domain resolution:}
 specifies the resolution of the LIS model output

 \var{LIS TWS output domain x-dimension size:}
 specifies the x-dimension size of the LIS model output

 \var{LIS TWS output domain y-dimension size:}
 specifies the y-dimension size of the LIS model output

 

 \begin{Verbatim}[frame=single]
GRACE raw data filename:    ../GRACE_tws/GRACE.CSR.LAND.RL05.DS.G200KM.nc
GRACE baseline starting year:      2004
GRACE baseline ending year:        2009
GRACE scale factor filename:       EXAMPLE
GRACE measurement error filename:  EXAMPLE
LIS TWS output format:                  "netcdf"
LIS TWS output methodology:             "2d gridspace"
LIS TWS output naming style:            "3 level hierarchy"
LIS TWS output map projection:          "latlon"
LIS TWS output nest index:              1
LIS TWS output directory:               ../OL_NLDAS/OUTPUT
LIS TWS output domain lower left lat:         25.0625
LIS TWS output domain lower left lon:        -124.9375
LIS TWS output domain upper right lat:        52.9375
LIS TWS output domain upper right lon:        -67.0625
LIS TWS output domain resolution (dx):          0.125
LIS TWS output domain resolution (dy):          0.125
 \end{Verbatim}

 
 \var{Simulated GRACE data directory:} specifies the directory
 containing the simulated GRACE observations.

 \var{Simulated GRACE configuration:} specifies the simulated GRACE
 configuration.
 Acceptable values are:

 \begin{tabular}{ll}
 Value         & Description \\
 ``default''   & default     \\
 ``follow-on'' & follow on   \\
 ``GRACE-2''   & GRACE-2     \\
 \end{tabular}

 \var{Simulated GRACE baseline starting year:}
 specifies the baseline starting year from which to establish the 
 simulated TWS climatology.

 \var{Simulated GRACE baseline ending year:}
 specifies the baseline ending year from which to establish the
 simulated TWS climatology.

 \var{LIS TWS output format:}
 specifies the output format of the LIS model output
 (binary/netcdf/grib1).

 \var{LIS TWS output methodology:}
 specifies the output methodology used in the LIS model output
 (1d tilespace/1d gridspace/2d gridspace).

 \var{LIS TWS output naming style:}
 specifies the output naming style used in the LIS model output 
 (3 level hierarchy/4 level hierarchy, etc.).

 \var{LIS TWS output nest index:}
 specifies the index of the nest used in the LIS model output.

 \var{LIS TWS output directory:}
 specifies the location of the LIS model output.

 \var{LIS TWS output map projection:} 
 specifies the map projection used in the LIS model output.

 For lat/lon projection:

 \var{LIS TWS output domain lower left lat:}
 specifies the lower left latitude of the LIS model output
 (if map projection is latlon).

 \var{LIS TWS output domain lower left lon:}
 specifies the lower left longitude of the LIS model output
 (if map projection is latlon).

 \var{LIS TWS output domain upper right lat:}
 specifies the upper right latitude of the LIS model output
 (if map projection is latlon).

 \var{LIS TWS output domain upper right lon:}
 specifies the upper right longitude of the LIS model output
 (if map projection is latlon).

 \var{LIS TWS output domain resolution (dx):}
 specifies the resolution (in degrees) along the latitude of the
 LIS model output (if map projection is latlon).

 \var{LIS TWS output domain resolution (dy):}
 specifies the resolution (in degrees) along the longitude of the
 LIS model output (if map projection is latlon).

 For Lambert and polar projections:

 \var{LIS TWS output domain lower left lat:}
 specifies the lower left latitude of the LIS model output

 \var{LIS TWS output domain lower left lon:}
 specifies the lower left longitude of the LIS model output

 \var{LIS TWS output domain true lat1:}
 specifies the true lat1 of the LIS model output

 \var{LIS TWS output domain true lat2:}
 specifies the true lat2 of the LIS model output

 \var{LIS TWS output domain standard lon:}
 specifies the standard longitude of the LIS model output

 \var{LIS TWS output domain resolution:}
 specifies the resolution of the LIS model output

 \var{LIS TWS output domain x-dimension size:}
 specifies the x-dimension size of the LIS model output

 \var{LIS TWS output domain y-dimension size:}
 specifies the y-dimension size of the LIS model output

 

 \begin{Verbatim}[frame=single]
Simulated GRACE data directory:         sim_grace
Simulated GRACE configuration:          default
Simulated GRACE baseline starting year: 2004
Simulated GRACE baseline ending year:   2009
LIS TWS output format:                  "netcdf"
LIS TWS output methodology:             "2d gridspace"
LIS TWS output naming style:            "3 level hierarchy"
LIS TWS output map projection:          "latlon"
LIS TWS output nest index:              1
LIS TWS output directory:               ../OL_NLDAS/OUTPUT
LIS TWS output domain lower left lat:         25.0625
LIS TWS output domain lower left lon:        -124.9375
LIS TWS output domain upper right lat:        52.9375
LIS TWS output domain upper right lon:        -67.0625
LIS TWS output domain resolution (dx):          0.125
LIS TWS output domain resolution (dy):          0.125
 \end{Verbatim}

 
 \var{NASA SMAP soil moisture data designation:} \attention{specifies what?}

 \var{NASA SMAP soil moisture observation directory:}
 specifies the location of the data directory containing the NASA 
 SMAP data.

 

 \begin{Verbatim}[frame=single]
NASA SMAP soil moisture data designation:
NASA SMAP soil moisture observation directory:
 \end{Verbatim}

 
 \var{SMOS NESDIS soil moisture observation directory:} specifies
 the location of the data directory containing the SMOS soil
 moisture retrievals from NOAA NESDIS.
 

 \begin{Verbatim}[frame=single]
SMOS NESDIS soil moisture observation directory:
 \end{Verbatim}


 
 \var{HYMAP river width map:}
 specifies the name of the HYMAP river width data file.

 \var{HYMAP river height map:}
 specifies the name of the HYMAP river height data file. 

 \var{HYMAP river roughness map:}
 specifies the name of the HYMAP river roughness data file. 

 \var{HYMAP floodplain height map:}
 specifies the name of the HYMAP floodplain height data file. 

 \var{HYMAP floodplain height levels:}
 specifies the number of the HYMAP floodplain height levels. 

 \var{HYMAP flow direction x map:}
 specifies the name of the x-flow direction data file.

 \var{HYMAP flow direction y map:}
 specifies the name of the y-flow direction data file.

 \var{HYMAP grid elevation map:}
 specifies the name of the grid elevation data file.

 \var{HYMAP grid distance map:}
 specifies the name of the grid distance data file.

 \var{HYMAP grid area map:}
 specifies the name of the grid area data file.

 \var{HYMAP drainage area map:} \attention{specifies what?}

 \var{HYMAP basin map:} \attention{specifies what?}

 \var{HYMAP runoff time delay map:}
 specifies the name of the runoff time delay data file. 

 \var{HYMAP runoff time delay multiplier map:}
 specifies the name of the runoff time delay multiplier data file.

 \var{HYMAP baseflow time delay map:}
 specifies the name of the baseflow time delay data file .

 \var{HYMAP reference discharge map:}
 specifies the name of the reference discharge data file.

 \var{HYMAP basin mask map:}
 specifies the name of the basin mask data file.

 \var{HYMAP params map projection:} \attention{specifies what?}

 \var{HYMAP params spatial transform:} \attention{specifies what?}

 \var{HYMAP river length map:}
 specifies the name of the river length data file.

 \var{HYMAP floodplain roughness map:}
 specifies the name of floodplain roughness map file.

 

 \begin{Verbatim}[frame=single]
HYMAP river width map:              ../HYMAP_parms/rivwth_Getirana_Dutra.bin
HYMAP river height map:             ../HYMAP_parms/rivhgt_Getirana_Dutra.bin 
HYMAP river roughness map:          ../HYMAP_parms/rivman_Getirana_Dutra.bin 
HYMAP floodplain roughness map:     ../HYMAP_parms/fldman.bin
HYMAP river length map:             ../HYMAP_parms/rivlen.bin
HYMAP floodplain height map:        ../HYMAP_parms/fldhgt.bin
HYMAP floodplain height levels:     10
HYMAP flow direction x map:         ../HYMAP_parms/nextx.bin
HYMAP flow direction y map:         ../HYMAP_parms/nexty.bin
HYMAP grid elevation map:           ../HYMAP_parms/elevtn.bin
HYMAP grid distance map:            ../HYMAP_parms/nxtdst.bin
HYMAP grid area map:                ../HYMAP_parms/grarea.bin
HYMAP drainage area map:
HYMAP basin map:
HYMAP runoff time delay map:        ../HYMAP_parms/kirpich.bin
HYMAP runoff time delay multiplier map: ../HYMAP_parms/trunoff.bin
HYMAP baseflow time delay map:      ../HYMAP_parms/tbasflw_45_amazon.bin
HYMAP reference discharge map:      ../HYMAP_parms/qrefer.bin
HYMAP basin mask map:               ../HYMAP_parms/mask_all.bin 
HYMAP params map projection:
HYMAP params spatial transform:
 \end{Verbatim}

 
 This section also outlines the domain specifications of the
 HYMAP parameter data.
 For the HYMAP parameters spatial transform option, only 'none' is
 supported at this time, and the user is required to input the HYMAP
 parameters at the grid and resolution of interest.

 If the map projection of parameter data is specified to be lat/lon,
 the following configuration should be used for specifying HYMAP data
 See Appendix~\ref{sec:d_latlon_example} for more details about
 setting these values.
 

 \begin{Verbatim}[frame=single]
HYMAP params spatial transform:      none
HYMAP params map projection:        latlon
HYMAP params lower left lat:        -59.9375
HYMAP params lower left lon:        -179.9375
HYMAP params upper right lat:        89.9375
HYMAP params upper right lon:       179.9375
HYMAP params resolution (dx):         0.125
HYMAP params resolution (dy):         0.125
 \end{Verbatim}


 
 \subsection{Artificial neural networks} \label{ssec:ann}

 \var{ANN input data sources:} \attention{specifies what?}

 \var{ANN mode (training/validation):} \attention{specifies what?}

 \var{ANN number of hidden neurons:} \attention{specifies what?}

 \var{ANN number of input data sources:} \attention{specifies what?}

 \var{ANN number of iterations:} \attention{specifies what?}

 \var{ANN number of parameters in each input source:} \attention{specifies what?}

 \var{ANN output data source:} \attention{specifies what?}

 \var{ANN training output file:} \attention{specifies what?}
 

 \begin{Verbatim}[frame=single]
ANN input data sources:
ANN mode (training/validation):
ANN number of hidden neurons:
ANN number of input data sources:
ANN number of iterations:
ANN number of parameters in each input source:
ANN output data source:
ANN training output file:
 \end{Verbatim}

 
 \subsubsection{GHCN} 

 \var{GHCN data directory:} \attention{specifies what?}

 \var{GHCN station file:} \attention{specifies what?}
 

 \begin{Verbatim}[frame=single]
GHCN data directory:
GHCN station file:
 \end{Verbatim}

 
 \subsubsection{LIS soil moisture output}

 \var{LIS soil moisture output timestep:} \attention{specifies what?}

 \var{LIS soil moisture output format:} \attention{specifies what?}

 \var{LIS soil moisture output methodology:} \attention{specifies what?}

 \var{LIS soil moisture output naming style:} \attention{specifies what?}

 \var{LIS soil moisture output map projection:} \attention{specifies what?}

 \var{LIS soil moisture output nest index:} \attention{specifies what?}

 \var{LIS soil moisture output directory:} \attention{specifies what?}

 For Lat/Lon projections:

 \var{LIS soil moisture domain lower left lat:}
 specifies the lower left latitude of the LIS model output

 \var{LIS soil moisture domain lower left lon:}
 specifies the lower left longitude of the LIS model output

 \var{LIS soil moisture domain upper right lat:}
 specifies the upper right latitude of the LIS model output

 \var{LIS soil moisture domain upper right lon:}
 specifies the upper right longitude of the LIS model output

 \var{LIS soil moisture domain resolution (dx):}
 specifies the resolution (in degrees) along the latitude of the
 LIS model output

 \var{LIS soil moisture domain resolution (dy):}
 specifies the resolution (in degrees) along the longitude of the
 LIS model output

 For Lambert and polar projections:

 \var{LIS soil moisture domain lower left lat:}
 specifies the lower left latitude of the LIS model output

 \var{LIS soil moisture domain lower left lon:}
 specifies the lower left longitude of the LIS model output

 \var{LIS soil moisture domain true lat1:}
 specifies the true lat1 of the LIS model output

 \var{LIS soil moisture domain true lat2:}
 specifies the true lat2 of the LIS model output

 \var{LIS soil moisture domain standard lon:}
 specifies the standard longitude of the LIS model output

 \var{LIS soil moisture domain resolution:}
 specifies the resolution of the LIS model output

 \var{LIS soil moisture domain x-dimension size:}
 specifies the x-dimension size of the LIS model output

 \var{LIS soil moisture domain y-dimension size:}
 specifies the y-dimension size of the LIS model output
 

 \begin{Verbatim}[frame=single]
LIS soil moisture output timestep:
LIS soil moisture output format:
LIS soil moisture output methodology:
LIS soil moisture output naming style:
LIS soil moisture output map projection:
LIS soil moisture output nest index:
LIS soil moisture output directory:
LIS soil moisture domain lower left lat:
LIS soil moisture domain lower left lon:
LIS soil moisture domain upper right lat:
LIS soil moisture domain upper right lon:
LIS soil moisture domain resolution (dx):
LIS soil moisture domain resolution (dy):
 \end{Verbatim}

 
 \subsubsection{MOD10A1} 

 \var{MOD10A1 data directory:} \attention{specifies what?}

 

 \begin{Verbatim}[frame=single]
MOD10A1 data directory:
 \end{Verbatim}

 
 \subsubsection{MODIS LST} 

 \var{MODIS LST data directory:} \attention{specifies what?}

 

 \begin{Verbatim}[frame=single]
MODIS LST data directory:
 \end{Verbatim}

