
\section{READ GRIB - Information and Instructions} \label{sec:readgrib_appendix}
Thanks to Kristi Arsenault for putting this together!\\

\textbf{Caveat}: \\
This is a package of subroutines to read GRIB-formatted data.  It is still under continuous development.  It won't be able to read every GRIB dataset you give it, but it will read a good many. \\
                                                                            
  - Kevin W. Manning \\
   NCAR/MMM        \\
   Summer 1998, and continuing \\

The main user interfaces are:    

\textbf{SUBROUTINE GRIBGET(NUNIT, IERR)}
 - Read a single GRIB record from UNIX file-descriptor NUNIT into array GREC. No unpacking of any header or data values is performed. 
                                                          
\textbf{SUBROUTINE GRIBREAD(NUNIT, DATA, NDATA, IERR)}
      - Read a single GRIB record from UNIX file-descriptor NUNIT, and unpack  
        all header and data values into the appropriate arrays.                
                                                                             
\textbf{SUBROUTINE GRIBHEADER(IERR) }                    
      - Unpack the header of a GRIB record                                 
                                                                            
\textbf{SUBROUTINE GRIBDATA(DATARRAY, NDAT)}
      - Unpack the data in a GRIB record into array DATARRAY               
                                                                             
\textbf{SUBROUTINE GRIBPRINT(ISEC)}
      - Print the header information from GRIB section ISEC.                  
                                                                            
\textbf{SUBROUTINE GET\_SEC1(KSEC1) }                                              
      - Return the header information from Section 1.                         
                                                                            
\textbf{SUBROUTINE GET\_SEC2(KSEC2) }
      - Return the header information from Section 2.                         
                                                                           
\textbf{SUBROUTINE GET\_GRIDINFO(IGINFO, GINFO) }
      - Return the grid information of the previously-unpacked GRIB header.   

\textbf{C-ROUTINE -- COPEN(UNIT, NUNIT, NAME, MODE, ERR, OFLAG)}
      - Opens the GRIB file for reading later by the other routines. 


\subsection{ SUBROUTINE GRIBGET (NUNIT, IERR)} 
  
    - Read a single GRIB record from UNIX file-descriptor NUNIT into array   
        GREC. No unpacking of any header or data values is performed.   \\
    - NOTE!:  Intrinsic parameter, ied, identifies type of GRIB edition of GRIB file trying to open.  Below are the codes (also see Table\ref{table:sec0}):

     -- If IED == 1, then GRIB Edition 1 has the size of the whole GRIB record right up front.\\
     -- If IED == 0, then GRIB Edition 0 does not include the total size, so we have to sum up  the sizes of the individual sections \\
     -- If IED $>$ 1, then STOP, if not GRIB Edition 0 or 1\\

\begin{itemize}
 \item[Input: ]
     NUNIT (integer):  C unit number to read from.  This should already be open. \item[Output:]      
     IERR (integer): Error flag, Non-zero means there was a problem with the read.    
                                                                           
\item[Side Effects:]                                                             
        The array GREC is allocated, and filled with one GRIB record.      
        The C unit pointer is moved to the end of the GRIB record just read. 
\end{itemize}


\subsection{SUBROUTINE GRIBREAD (NUNIT, DATA, NDATA, IERR) }
   
   - Read a single GRIB record from UNIX file-descriptor, NUNIT, unpack  
        all header and data values into the appropriate arrays, and fill the allocatable array,   
        DATARRAY(:).   
\begin{itemize}        
 \item[ Input:]
    NUNIT (integer):  C Unit to read from.                                             \\
    NDATA (integer):  Size of array DATA (Should be >= NDAT as computed herein.)       
                                                                             
  \item[ Output:  ]                                                                   
    DATA (real):  The unpacked data array (dimension size of NDATA)    \\
    IERR(integer):  Error flag, non-zero means there was a problem.                   
                                                                             
  \item[Side Effects: ]                                                              
     Header arrays SEC0, SEC1, SEC2, SEC3, SEC4, XEC4, INFOGRID and         
        INFOGRID are filled.                    \\                               
    The BITMAP array is filled.                 \\                           
    The C unit pointer is advanced to the end of the GRIB record.  \\

\end{itemize}

\subsection{SUBROUTINE GRIBHEADER (IERR)}
                                         
Unpack the header of a GRIB record \\
IERR non-zero means there was a problem unpacking the grib header\\
IERR (integer)\\

\subsection{ SUBROUTINE GRIBDATA (DATARRAY, NDAT)    }
                                 
      - Read and unpack the data in a GRIB record into array, DATARRAY

\begin{itemize}
 \item[Input:]                                                    
    NDAT (integer):  The size of the data array we expect to unpack.                  
                                                                            
\item[ Output:]                                                        
    DATARRAY (real):  The unpacked data from the GRIB record (dimension size of NDAT)  
\item[ Side Effects:] 
   - STOP -- if it cannot unpack the data.  
\end{itemize}



\subsection{ SUBROUTINE GRIBPRINT (ISEC)   }

 Print the header information from GRIB section ISEC.\\
 ISEC Information can be found in the section\ref{grid_desc}\\

\subsection{SUBROUTINE GET\_SEC1 (KSEC1)}

      - Return the header information from GRIB Section 1 (see\ref{grid_desc}, TABLE~\ref{table:sec1}). \\
      - Return the GRIB Section 1 header information, which has already been 
         unpacked by subroutine GRIBHEADER. \\
      - KSEC1 (integer :: dimension(100))     \\  
                                                                    

\subsection{ SUBROUTINE GET\_SEC2 (KSEC2) }

      - Return the header information from GRIB Section 2 (see\ref{grid_desc}, TABLE~\ref{table:sec2}).\\
      - Return the GRIB Section 2 header information, which has already been
          unpacked by subroutine GRIBHEADER.    \\
      - KSEC2 (integer :: dimension(10))         \\
                                                                           
\subsection{ SUBROUTINE GET\_GRIDINFO (IGINFO, GINFO) }

      - Return the grid information of the previously-unpacked GRIB header.\\
      - IGINFO (integer :: dimension(40))\\
      - GINFO (real :: dimension(40))\\
** NOTE  IGINFO and GINFO contain equivalent information, except that IGINFO is the integer form and GINFO is the real form.\\
 
\subsection{C-ROUTINE  COPEN (UNIT, NUNIT, NAME, MODE, ERR, OFLAG)}

      - Opens the GRIB file for reading later by the other subroutines \\

  	UNIT    = Fortran unit number  (integer) \\
    	
	NUNIT = UNIX file descriptor associated with 'unit' (integer) \\
 
 	NAME  = UNIX file name (character (len=120) )\\

  	MODE  = 0 : write only - file will be created if it doesn't exist, 
                           - otherwise will be rewritten (integer)\\
       		  = 1 : read only\\
         		  = 2 : read/write\\

	 ERR    = 0 : no error opening file  (integer)\\
       	       != 0 : Error opening file\\
         
	 OFLAG = 0 : file name printed (no errors printed) (integer)\\
          		  $>$ 0 : file name printed and errors are printed\\
          		  $<$ 0 : no print at all (not even erros) \\

\subsection{SEC Header Array Information Tables}\label{grid_desc}

Please refer to \textsl{http://www.wmo.ch/web/wwww/WDM/Guides/Guide-binary-2.html} for additional GRIB1 header information\\

\begin{table}[H]
\centering
\caption{SEC0: GRIB Header Section 0 information}
\vspace{0.2in}
\begin{tabular}{l|l}
\hline
\hline
Number   & Description\\
\hline
\hline
1 & Length of a complete GRIB record \\
2 & Grib Edition Number \\
\hline
\hline
\end{tabular}
\label{table:sec0}
\end{table}

\begin{table}[H]
\centering
\caption{SEC1: GRIB Header Section 1 information}
\vspace{0.2in}
\begin{tabular}{l|l}
\hline
\hline
Octet  & PDS content\\
\hline
 1-3  &  Length of GRIB section 1 (3 bytes)   \\
    4  &  Parameter Table Version number    \\                       
    5  &  Center Identifier               \\                         
    6  &  Generating process Identifier     \\                                   
    7  &  Grid ID number for pre-specified grids. \\                       
    8  &  Binary bitmap flag:                  \\
    9  &  Parameter ID Number and Units (ON388 Table 2) \\                           
   10 &  Indicator of level type or layer (ON388 Table 3) \\                                    
   11 &  Level value (height or pressure), of the top value of a layer  \\                          
   12 &  Level value, but for bottom value of a layer ( 0 if NA ??) \\                         
   13 &  Year (00-99)      \\                                             
   14 &  Month (01-12)       \\                                           
   15 &  Day of the month (01-31) \\                                       
   16 &  Hour (00-23)           \\                                         
   17 &  Minute (00-59)           \\                                      
   18 &  Forecast time unit: (ON388 Table 4) \\                            
   19 &  Time period 1 (Number of Time Units  Given in Octet 18) \\                                              
   20 &  Time period 2 or time interval between successive analyses\\                                                 
   21 &  Time range indicator (ON833 Table 5)   \\                         
22-23 & Number included in average when Octet 21 (Table 5) \\
      & indicates average or accumulation (otherwise set to 0) \\                                 
   24 &  Number missing from averages or accumulations    \\                       
   25 &  Century (Years 1999 and 2000 are century 20, 2001 is century 21)\\
   26 &  Sub-center identifier             \\                            
27-28 &  Decimal scale factor D.  Negative value indicates \\ 
      & setting high order bit in Octet 27 to 1 (``on''). \\                                  
   29 &  Is there a GDS (0=no, 1=yes; bit 1 of sec1(6))  Refer to Octet 8 above       \\
   30 &  Is there a BMS (0=no, 1=yes; bit 2 of sec1(6))  Refer to Octet 8 above \\

\hline
\hline
\end{tabular}
\label{table:sec1}
\end{table}

\setlongtables
\begin{longtable}{l|l|l}
\caption{SEC2: GRIB Header Section 2 information}\\
\hline
\hline
Octet & GDS Content & \\
\hline
     1-3 &  Length of GRIB Section 2 (in octets)    &  \\
       4  &  Number of vertical-& \\
          & coordinate parameters &  \\            
       5  &  Starting-point of the list of & \\
          &  vertical-coordinate parameters & \\
       6  &  Data-representation type & \\
          &   (i.e., grid type)  See GRIB Table 6 &  \\  
           &      0 = Latitude/Longitude grid &  \\                       
           &      3 = Lambert-conformal grid. &  \\                         
           &      5 = Polar-stereographic grid. &  \\    
\hline
\multicolumn{2}{c}{if(sec2(4)==0) then Lat/lon grid} & \\
INFOGRID & Octet & GDS Content \\
\hline
     1 &    7-8  &  I Dimension of the grid  \\                                 
     2 &  9-10  &  J Dimension of the grid   \\                                
     3 &      11-13  &  Starting Latitude of the grid. \\                           
     4 &      14-16  &  Starting Longitude of the grid. \\                          
     5 &     17  &  Resolution and component flags.  \\                         
     6 &      18-20  &  Ending latitude of the grid. \\                             
     7 &      21-23  &  Ending longitude of the grid.\\                             
     8 &      24-25  &  Longitudinal increment.  \\                                 
     9 &      26-27  &  Latitudinal increment.     \\                               
    10 &    28  &  Scanning mode (bit 3 from Table 8) \\                        
    21 &    28  &  Iscan sign (+1/-1) (bit 1 from Table 8) \\                  
    22 &    28  &  Jscan sign (+1/-1) (bit 2 from Table 8) \\
\hline
\multicolumn{2}{c}{if(sec2(4)==1) then mercator grid }& \\
INFOGRID & Octet & GDS Content \\
\hline
      1 &             7-8   &  I Dimension of the grid   \\
      2 &            9-10   &  J Dimension of the grid    \\                               
      3 &          11-13   &  Starting Latitude of the grid.\\                            
      4 &          14-16   &  Starting Longitude of the grid.\\                          
      5 &        17   &  Resolution and component flags.   \\                        
      6 &          18-20   &  Ending latitude of the grid. \\                             
      7 &          21-23   &  Ending longitude of the grid. \\                            
      8 &          24-26   &  LATIN- The latitude(s) at which \\
        &                  & Mercator projection \\
        &                  & cylinder intersects the earth  \\                       
      9 &               27   &  Reserved (set to 0)       \\                    
     10 &               28   &  Scanning mode (bit 3 from Table 8) \\
     11 &                      &   True Lat \\ 
     21 &          29-31   &  Iscan sign (+1/-1) (bit 1 from Table 8) \\                  
     22 &          32-34   &  Jscan sign (+1/-1) (bit 2 from Table 8)   \\                
\hline
\multicolumn{2}{c}{if(sec2(4)==3) then Lambert Conformal grid} & \\
INFOGRID & Octet & GDS Content \\
\hline
      1 &          7-8  &  I Dimension of the grid   \\
      2 &        9-10  &  J Dimension of the grid    \\                               
      3 &      11-13  &  Starting Latitude of the grid.\\                            
      4 &      14-16  &  Starting Longitude of the grid. \\                          
      5 &           17  &  Resolution and component flags. \\                   
      6 &      18-20  &  Center longitude of the projection. \\                      
      7 &      21-23  &  Grid-spacing in the I direction   \\                        
      8 &      24-26  &  Grid-spacing in the J direction  \\                         
      9 &           27  &  Projection center         \\                                
     10 &          28  &  Scanning mode (bit 3 from Table 8) \\                       
     11 &     29-31  &  First TRUELAT value.     \\                                 
     12 &     32-34  &  Second TRUELAT value.      \\                               
     13 &     35-37  &  Latitude of the southern pole \\                          
     14 &     38-40  &  Longitude of the southern pole \\                        
     21 &          41  &  Iscan sign (+1/-1) (bit 1 from Table 8) \\                 
     22 &          42  &  Jscan sign (+1/-1) (bit 2 from Table 8) \\              

\hline
\multicolumn{2}{c}{if(sec2(4)==4) then Gaussian grid} & \\
INFOGRID & Octet & GDS Content \\
\hline
    1 &          7-8  &  I Dimension of the grid   \\                                
     2 &        9-10  &  J Dimension of the grid     \\                              
     3 &      11-13  &  Starting Latitude of the grid. \\                           
     4 &      14-16  &  Starting Longitude of the grid.  \\                         
     5 &           17  &  Resolution and component flags. \\                          
     6 &      18-20  &  Ending latitude of the grid.   \\                           
     7 &      21-23  &  Ending longitude of the grid. \\                            
     8 &      24-25  &  Longitudinal increment.  \\                                 
     9 &      26-27  &  Latitudinal increment.     \\                               
    10 &          28  &  Scanning mode (bit 3 from Table 8) \\                       
    21 &          28  &  Iscan sign (+1/-1) (bit 1 from Table 8) \\                  
    22 &          28  &  Jscan sign (+1/-1) (bit 2 from Table 8) \\  
\hline
\multicolumn{2}{c}{if(sec2(4)==5) then Polar stereographic grid }& \\
INFOGRID & Octet & GDS Content \\
\hline
       1 &          7-8  &  I Dimension of the grid  \\
       2 &        9-10  &  J Dimension of the grid   \\                                
       3 &      11-13  &  Starting Latitude of the grid.\\                            
       4 &      14-16  &  Starting Longitude of the grid. \\                          
       5 &         17 &  Resolution and component flags.  \\                         
       6 &        18-20 &  Center longitude of the projection. \\                      
       7 &        21-23 &  Grid-spacing in the I direction \\          
       8 &        24-26 &  Grid-spacing in the J direction \\                          
       9 &             27 &  Projection center      \\                                   
      10 &            28 &  Scanning mode (bit 3 from Table 8) \\                       
      21 &            29 &  Iscan sign (+1/-1) (bit 1 from Table 8) \\                  
      22 &            30 &  Jscan sign (+1/-1) (bit 2 from Table 8)\\
\hline
\multicolumn{2}{c}{if(sec2(4)==50) then Spherical Harmonic Coefficients} & \\
INFOGRID & Octet & GDS Content \\
\hline
        1&       7-8  &  J-pentagonal resolution parameter    \\
        2&     9-10  &  K-pentagonal resolution parameter  \\                       
        3&   11-12  &  M-pentagonal resolution parameter   \\                      
        4&         13  &  Spectral representation \\
         &               & type (ON388 Table 9) \\             
        5&         14  &  Coefficient storage mode (ON388 Table 10) \\                
         &             15-32 &  Set to 0 (reserved) \\
\hline
\hline
\label{table:sec2}
\end{longtable}



\subsection{Additional information for setting up the READ\_GRIB routines for use on Linux Machines}

	A few steps were taken to modify the original READ\_GRIB routines to make them more compatible with Absoft, Lahey95, and other Linux (32-bit and 64-bit) based compilers.  Here is a list of those steps:\\

Replaced the extensions of each *.F file with *.F90.\\

In the C-routine, cio.c, the following lines of code were added or modified:
 -- Line 19   Added ``$||$ defined(ABSOFT)and $||$ defined(LAHEY)''\\

In the Makefile, the following lines of code were added or modified:\\
 -- Line 19   Added ``.F90''  to .SUFFIXES rule \\
 -- Line 34-35 --  Added ``@echo make absoft'' and ``@echo make lahey''; resp.   \\
 -- Line 71 (and following lines)  Added flags and compiler names for ABSOFT:\\

\begin{verbatim}
absoft:
       $(MAKE) $(LIBTARGET) \
       "FC = f90" \
       "FCFLAGS = -O -YEXT_NAMES=LCS -B108 -YCFRL=1 -YDEALLOC=ALL -     
       DHIDE_SHR_MSG -DNO_SHR_VMATH -DABSOFT -DLITTLE_ENDIAN -DBIT -DBIT32" \
       "CC  = gcc" \
       "CCFLAGS = -O -Wall -DABSOFT -DLITTLE_ENDIAN -DG_ENABLE_DEBUG=1" \
       "CPP = /lib/cpp" \
               "CPPFLAGS = -C -P -DBIT32"

lahey:
        $(MAKE) $(LIBTARGETS) \
        "FC = lf95" \
        "FCFLAGS = -O -DBIT32 -DLINUX -DLAHEY -DLITTLE_ENDIAN"\
        "CC  = cc" \
        "CCFLAGS = -O -DUSE_GCC -DLAHEY -DLITTLE_ENDIAN" \
        "CPP = /lib/cpp" \
        "CPPFLAGS = -C -P "

\end{verbatim}

-- Line 90   Changed ``.F.o'' to ``.F90.o''\\  

-- Line 91   Removed ``-d'' from the rule\\


\subsection{ Example of Fortran code that calls READ\_GRIB routines}

\begin{verbatim}
!-- Initialize certain variables and parameters of GRIB file:
    nunit = 10           ! Fortran unit number
    ufn = nunit + 1   ! UNIX file descriptor associated with "nunit"
    datarray = 0

!-- Open INPUT GRIB File:
    call copen (nunit, ufn, trim(input_file)//char(0), 1, iret, 1)
     print *, " ** Open File Code: ", iret

     if ( iret > 0) then     ! Return File Error Code Number - IF FAILED TO OPEN!
        write(*,*) "STOPPING ROUTINE -- FILE NOT OPENED DUE TO CODE # :: ", iret
        stop
     end if

!-- Read GRIB file: 
    call gribread ( ufn, datarray, ndata, ierr )

     if ( ierr > 0) then   ! Return File Error Code Number - IF FAILED TO READ!
        write(*,*) "STOPPING ROUTINE -- FILE NOT READ DUE TO CODE # :: ", ierr
        stop
     end if

!Print GRIB Header Information
    do isec = 0, 2
       call gribprint (isec)
    end do

    do i = 1, ndata
       if (datarray(i)  >0 ) then
       print *, i, datarray(i)
       end if
    end do
\end{verbatim}
