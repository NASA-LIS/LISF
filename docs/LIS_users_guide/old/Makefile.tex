
\section{Makefile} \label{sec:makefile}

\begin{verbatim}
# Set up special characters
include ./configure.lis

null  :=
space := $(null) $(null)
doctool :=../utils/docsgen.sh

# Check for directory in which to put executable
ifeq ($(MODEL_EXEDIR),$(null))
MODEL_EXEDIR := .
endif

# Check for name of executable
ifeq ($(EXENAME),$(null))
EXENAME := LIS
endif

# Check if COUPLED is defined in "LIS_misc.h"
# Ensure that it is defined and not just "undef COUPLED" set in file
CPLD := $(shell grep COUPLED LIS_misc.h)
ifeq ($(findstring define,$(CPLD)), define)
  CPLD := TRUE
else
  CPLD := FALSE
endif

# Determine platform
UNAMES := $(shell uname -s)
UMACHINE := $(shell uname -m)

# Load dependency search path.
dirs := . $(shell cat Filepath)
ifeq ($(CPLD), FALSE)
dirs := $(dirs) ../offline
endif
# Set cpp search path, include netcdf
cpp_dirs := $(dirs) 
cpp_path := $(foreach dir,$(cpp_dirs),-I$(dir)) # format for command line

# Expand any tildes in directory names. Change spaces to colons.
VPATH    := $(foreach dir,$(cpp_dirs),$(wildcard $(dir)))
VPATH    := $(subst $(space),:,$(VPATH))

#------------------------------------------------------------------------
# Primary target: build the model
#------------------------------------------------------------------------

# Get list of files and determine objects and dependency files
FIND_FILES = $(wildcard $(dir)/*.F $(dir)/*.f $(dir)/*.F90 $(dir)/*.f90 $(dir)/*.c $(dir)/*.f)
FILES      = $(foreach dir, $(dirs),$(FIND_FILES))
SOURCES   := $(sort $(notdir $(FILES)))
DEPS      := $(addsuffix .d, $(basename $(SOURCES)))
OBJS      := $(addsuffix .o, $(basename $(SOURCES)))
DOCS      := $(addsuffix .tex, $(basename $(SOURCES)))

$(MODEL_EXEDIR)/$(EXENAME): $(OBJS) version.o
	 $(FC) -o $@ $(OBJS) version.o $(FOPTS) $(LDFLAGS)

LIBTARGET    =  lislib
TARGETDIR    =  ./
$(LIBTARGET) : 
		gmake -i -r lis_contrib version.o                  ; \
		$(AR) lislib.a $(OBJS) version.o ; \

LIBTARGET    =  explis
TARGETDIR    =  ./
$(LIBTARGET) : 
		gmake -i -r lis_contrib version.o                  ; \
		$(AR) -ru ../../main/libwrflib.a $(OBJS) version.o ; \

LIBTARGET    =  gcelis
TARGETDIR    =  ./
$(LIBTARGET) : 
		gmake -i -r lis_contrib version.o                  ; \
		$(AR) lislib.a $(OBJS) version.o ; \

lis_contrib : $(OBJS)

debug: $(OBJS)
	echo "FFLAGS: $(FFLAGS)"
	echo "LDFLAGS: $(LDFLAGS)"
	echo "OBJS: $(OBJS)"

#***********************************************************************
#********** Architecture-specific flags and rules***********************
#***********************************************************************

FFLAGS      += $(cpp_path)

.SUFFIXES:
.SUFFIXES: .f .f90 .F90 .c .o

.f.o:
	$(FC77) $(FFLAGS77) $<
.F90.o:
	$(FC) $(FFLAGS) $<
.f90.o:
	$(FC) $(FFLAGS) $<
.c.o:
	$(CC) -c $(cpp_path) $(CFLAGS) $<
version.o: version.F90
	$(FC) $(FFLAGS) $<
	rm $<

version.F90:
	@echo "Building version file $@"
	sh build_version.sh


RM := rm
# Add user defined compiler flags if set, and replace FC if USER option set.
FFLAGS  += $(USER_FFLAGS)
ifneq ($(USER_FC),$(null))
FC := $(USER_FC)
endif

clean:
	$(RM) -f *.o *.mod *.stb  $(MODEL_EXEDIR)/$(EXENAME)
realclean:
	$(RM) -f *.o *.d *.mod *.stb  $(MODEL_EXEDIR)/$(EXENAME)
doc:
	$(doctool) 

#------------------------------------------------------------------------
#!!!!!!!!!!!!!!!!DO NOT EDIT BELOW THIS LINE.!!!!!!!!!!!!!!!!!!!!!!!!!!!!
#------------------------------------------------------------------------
# These rules cause a dependency file to be generated for each source
# file.  It is assumed that the tool "makdep" (provided with this
# distribution in clm2/tools/makdep) has been built and is available in
# the user's $PATH.  Files contained in the clm2 distribution are the
# only files which are considered in generating each dependency.  The
# following filters are applied to exclude any files which are not in
# the distribution (e.g. system header files like stdio.h).
#
#  1) Remove full paths from dependencies. This means gnumake will not break
#     if new versions of files are created in the directory hierarchy
#     specified by VPATH.
#
#  2) Because of 1) above, remove any file dependencies for files not in the
#     clm2 source distribution.
#
# Finally, add the dependency file as a target of the dependency rules.  This
# is done so that the dependency file will automatically be regenerated
# when necessary.
#
#     i.e. change rule
#       make.o : make.c make.h
#       to:
#       make.o make.d : make.c make.h
#------------------------------------------------------------------------
DEPGEN := ./MAKDEP/makdep -s F -s f90
%.d : %.c
	@echo "Building dependency file $@"
	@$(DEPGEN) -f $(cpp_path) $< > $@
%.d : %.f
	@echo "Building dependency file $@"
	@$(DEPGEN) -f $(cpp_path) $< > $@
%.d : %.F90
	@echo "Building dependency file $@"
	@$(DEPGEN) -f $(cpp_path) $<  > $@
%.d : %.f90
	@echo "Building dependency file $@"
	@$(DEPGEN) -f $(cpp_path) $<  > $@
%.d : %.F
	@echo "Building dependency file $@"
	@$(DEPGEN) -f $(cpp_path) $< > $@
#
# if goal is clean or realclean then don't include .d files
# without this is a hack, missing dependency files will be created
# and then deleted as part of the cleaning process
#
INCLUDE_DEPS=TRUE
ifeq ($(MAKECMDGOALS), realclean)
 INCLUDE_DEPS=FALSE
endif
ifeq ($(MAKECMDGOALS), clean)
 INCLUDE_DEPS=FALSE
endif

ifeq ($(INCLUDE_DEPS), TRUE)
-include $(DEPS)
endif

\end{verbatim}

